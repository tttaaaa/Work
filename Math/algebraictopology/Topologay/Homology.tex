%===============
%一行目に必ず必要
%文章の形式を定義
%===============
\documentclass{ujarticle}
%===============
%パッケージの定義、必要か不明
%===============
%この下4つを加えることで、mathbbが機能した
\usepackage{amsthm}
\usepackage{amsmath}
\usepackage{amssymb}
\usepackage{amsfonts}
%可換図式用パッケージ
\usepackage{amscd}
\usepackage[all]{xy}
\usepackage{tikz-cd}
%リンク用パッケージ
\usepackage[dvipdfmx]{hyperref}
%複数行コメント
%\usepackage{comment}

%タイトルデータ
\title{Definition of Homology}
\author{ta}


%===============
%定理環境の設定
%セクション毎
%===============
\newtheorem{thm}{Theorem}[section]
\newtheorem{dfn}[thm]{Definition}
\newtheorem{prop}[thm]{Propostion}
\newtheorem{lem}[thm]{Lemma}
\newtheorem{ex}[thm]{Example}
\newtheorem*{prob}{Problem}
\newtheorem*{rem}{Remark}
\newtheorem{prf}{Proof}

\begin{document}

% タイトルを出力
\maketitle
\section{Introduction}
\label{sec:Introduction}

Homologyの定義と性質について概説する。
本資料のメインテーマは胞体複体のHomologyが特異Homologyと一致することの証明である。
証明を理解し、実際に胞体複体のHomologyの計算ができれば、Homologyの初心者ではなくなっているであろう。
今のところに、以下について説明する予定である。
\begin{itemize}
  \setlength{\parskip}{0cm} % 段落間
  \setlength{\itemsep}{0cm} % 項目間
  \item Singular Homologyの定義
  \item Singular Homologyの基本的な性質
  \item 空間対のHomology
  \item 胞体腹体/単体複体のHomology
  \item Persistent Homology
\end{itemize}

位相空間の基本的性質についても、よく使うものは事前にまとめる予定である。

\section{Singular Homologyの定義}
\label{sec:Singular Homologyの定義}

n次元単体の定義
特異Chainの定義
特異Homologyの定義

\section{Singular Homologyの基本的な性質}
\label{sec:Singular Homologyの基本的な性質}
0次元Homology
Mayor-Vietris

\section{胞体複体のHomology}
\label{sec:胞体複体のHomology}
胞体複体のHomologyを定義し、これが実際に特異Homologyと一致していることをみる。

まず、胞体、および、胞体複体について定義する。

\begin{dfn}
  $X$が$n$次元胞体とは、
\end{dfn}





\begin{thebibliography}{数字}
  \bibitem{K} Elliptic Curves.・Anthony W. Knapp
  \bibitem{M} Modular Functions and Modular Forms・J. S. Milne
\end{thebibliography}

\end{document}
