%===============
%一行目に必ず必要
%文章の形式を定義
%===============
\documentclass{ujarticle}
%===============
%パッケージの定義、必要か不明
%===============
%この下4つを加えることで、mathbbが機能した
\usepackage{amsthm}
\usepackage{amsmath}
\usepackage{amssymb}
\usepackage{amsfonts}
%可換図式用パッケージ
\usepackage{amscd}
\usepackage[all]{xy}
\usepackage{tikz-cd}
%リンク用パッケージ
\usepackage[dvipdfmx]{hyperref}
%複数行コメント
%\usepackage{comment}


%===============
%定理環境の設定
%セクション毎
%===============
\newtheorem{thm}{Theorem}[section]
\newtheorem{dfn}[thm]{Definition}
\newtheorem{prop}[thm]{Propostion}
\newtheorem{lem}[thm]{Lemma}
\newtheorem{ex}[thm]{Example}
\newtheorem*{prob}{Problem}
\newtheorem*{rem}{Remark}
\newtheorem{prf}{Proof}

\begin{document}
ホモロジーの基礎について概説する。

\section{特異ホモロジーの定義}
\label{sec:特異ホモロジーの定義}

\section{ホモロジーの基本性質}
\label{sec:ホモロジーの基本性質}

\section{ホモトピー同値によるホモロジー群の不変性}
\label{sec:ホモトピー同値によるホモロジー群の不変性}

\section{マイヤーヴィエトリス完全列の証明}
\label{sec:マイヤーヴィエトリス完全列の証明}

\section{対のホモロジーと切除定理}
\label{sec:対のホモロジーと切除定理}

\section{胞体複体のホモロジー}
\label{sec:section label}




\end{document}
