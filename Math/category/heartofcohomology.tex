%===============
%一行目に必ず必要
%文章の形式を定義
%===============
\documentclass{ujarticle}
%===============
%パッケージの定義、必要か不明
%===============
%この下4つを加えることで、mathbbが機能した
\usepackage{amsthm}
\usepackage{amsmath}
\usepackage{amssymb}
\usepackage{amsfonts}
%可換図式用パッケージ
\usepackage{amscd}
\usepackage[all]{xy}
\usepackage{tikz-cd}
%リンク用パッケージ
\usepackage[dvipdfmx]{hyperref}
%複数行コメント
%\usepackage{comment}

%タイトルデータ
\title{Complex Multiplication}
\author{take}
\date{2016/October}


%===============
%定理環境の設定
%セクション毎
%===============
\newtheorem{thm}{Theorem}[section]
\newtheorem{dfn}[thm]{Definition}
\newtheorem{prop}[thm]{Propostion}
\newtheorem{lem}[thm]{Lemma}
\newtheorem{ex}[thm]{Example}
\newtheorem*{prob}{Problem}
\newtheorem*{rem}{Remark}
\newtheorem{prf}{Proof}

\begin{document}
コホモロジーのころろでの理解をまとめる。


\section{コホモロジーはすべて導来関手}
\label{sec:コホモロジーはすべて導来関手}

コホモロジーと言われるものはすべて、あるアーベル圏から、アーベル圏への導来関手として実現される。

その意味を導来関手を定義し、その性質をみることで確認するのが本章である。具体的には以下を説明する。
\begin{itemize}
  \item 導来関手の定義
  \item 定義から圏論的に存在するならば一意であること
 \item アーベル圏からアーベル圏への左完全な加法的関手に対し、導来関手がたた一つ存在すること
\item ホモトピックな射は導来関手は導来関手に対し、同じ射を定めること。
\item その他用語の定義
\end{itemize}
なお、直接的な計算は本章では行わない。

興味が有る場合は群のコホモロジー、あるいはガロアコホモロジーについて調べよ。





\end{document}
