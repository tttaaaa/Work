%===============
%一行目に必ず必要
%文章の形式を定義
%===============
\documentclass{ujarticle}
%===============
%パッケージの定義、必要か不明
%===============
%この下4つを加えることで、mathbbが機能した
\usepackage{amsthm}
\usepackage{amsmath}
\usepackage{amssymb}
\usepackage{amsfonts}
%可換図式用パッケージ
\usepackage{amscd}
\usepackage[all]{xy}
\usepackage{tikz-cd}
%リンク用パッケージ
\usepackage[dvipdfmx]{hyperref}
%複数行コメント
%\usepackage{comment}

%タイトルデータ
\title{Modular Formとその周辺}
\author{take}
\date{2016/August}


%===============
%定理環境の設定
%セクション毎
%===============
\newtheorem{thm}{Theorem}[section]
\newtheorem{dfn}[thm]{Definition}
\newtheorem{prop}[thm]{Propostion}
\newtheorem{lem}[thm]{Lemma}
\newtheorem{ex}[thm]{Example}
\newtheorem*{prob}{Problem}
\newtheorem*{rem}{Remark}
\newtheorem{prf}{Proof}

\begin{document}
\section{abelian categoryにおけるKernelの随伴関手}
\label{sec:abelian categoryで、Kernelの随伴関手}
Kernelの随伴関手を定義する。以下の順序にて説明する。
\begin{enumerate}
  \item abelian category $ \mathcal{C}$に対し、随伴で移り合うcategory $\mathcal{C}^*$を定義する。
  \item Kernelとその随伴になる関手を定義する。
  \item 実際に随伴となっていることを確認する。
\end{enumerate}

\subsection{$\mathcal{C}^*$の定義}


$\mathcal{C}$をabelian categoryとする。
$ \mathcal{C}^*$を以下で定義する。
\begin{enumerate}
  \item $Obj(\mathcal{C}^*)$はある$Y,Z \in \mathcal{C}$が存在し、$f \in Hom_{\mathcal{C}}(Y,Z)$
  となる$f$全体
  \item $f:Y \to Z$,$g:Y' \to Z'$に対し、$\tau \in \mathrm{Hom}(f,g)$は
  $\tau_Y:Y \to Y',\tau_Z:Z \to Z'$であって、$g \circ \tau_Y = \tau_Z \circ f$を満たすもの全体
\end{enumerate}
\begin{rem}
 $ \mathcal{C}^* $はabelian category(のはず)。
\end{rem}

\subsection{Kernelとその随伴関手の定義}
\label{sub:Kernelとその随伴関手の定義}

(コホモロジーのこころのKerの定義がよくわからなかったので)$ \mathcal{C} $の射$f:Y \to Z$のKernelを以下で定義する。
任意の$f \circ g = 0$となる$g:X \to Y$に対し、以下が可換になる射$X \to \mathrm{Ker}f$がただ一つ存在するような$(\mathrm{Ker}f,i)$の組のことを$f$のKernelという。

\xymatrix{ \ar@{}[rd]|{\circlearrowright}
X   \ar[rr] \ar@{.>}[d]& & Y \ar[r]^{f} & Z \\
\mathrm{Ker}f  \ar[urr]^{i} && & }
\begin{rem}
 TeX力がないので、いい可換図式がかけません。TeXGodがいたら、教えてください。
\end{rem}

これは、$\mathrm{Hom}_{\mathcal{C}}(X,\mathrm{Ker}f)$と$f$が誘導する準同型
$f^*:\mathrm{Hom}_{\mathcal{C}}(X,Y) \to \mathrm{Hom}_{\mathcal{C}}(X,Z)$のKernelの間に自然な同型があることを意味している。

functor $\mathrm{Ker}: \mathcal{C}^* \to \mathcal{C}$を以下で定義する。
$f: Y \to Z$に対し、関手Kerの$f$の像をabelian category$\mathcal{C}$のKer$f$とする。
$ \mathcal{C}^* $上の$f:Y \to Z$,$g:Y' \to Z'$に対する射$(\tau_Y,\tau_Z)$に対し、Ker$f$からKer$g$への射を以下で定める。

\xymatrix{ \ar@{}[rd]|{\circlearrowright} 
\mathrm{Ker}f   \ar@{.>}[d] \ar[r]^{i_f} &  Y \ar@{}[rd]|{\circlearrowright} \ar[r]^{f} \ar[d]^{\tau_Y}  & Z \ar[d]^{\tau_Z}  \\
\mathrm{Ker}g  \ar[r]^{i_g} & Y' \ar[r]^{g} & Z' }

これは$f \circ i_f$が0射になり、図式の可換性から、$g \circ \tau_Y \circ i_f$が0射となる。よって、Kernelのuniversalityから$\mathrm{Ker}f$から$\mathrm{Ker}g$の射がただひとつ定まるので、Well-definedとなる。


Kernelの随伴関手$F:\mathcal{C} \to \mathcal{C}^*$を定義する。
$X \in \mathcal{C}$に対し、$F(X)=0_X:X \to 0$で定める。
$f:Y \to Z$に対し、$F(f)= (f,\tau_{0})$と定める。

\xymatrix{ \ar@{}[rd]|{\circlearrowright} 
Y  \ar[r]^{0} \ar[d]^{f}  & 0 \ar[d]^{\tau_0}  \\
Z \ar[r]^{0} & 0 }


\subsubsection{随伴関手になることの確認}
\label{subs:随伴関手になることの確認}

$\mathrm{Hom}(X,\mathrm{Ker}f) \sim \mathrm{Hom}(F(X),f)$を示す。
$g \in \mathrm{Hom}(X,Y)$を、$(g.0) \in \mathrm{Hom}(F(X),f)$となるようにとる。
すると、Kernelのuniversalityより以下の可換図式が成りたつような射$g':X \to \mathrm{Ker}f$がただひとつ存在する。よって、示された。

\xymatrix{ \ar@{}[rd]|{\circlearrowright} 
  X \ar@{.>}[d] \ar[r]^{i_X} &  X \ar@{}[rd]|{\circlearrowright} \ar[r]^{0} \ar[d]^{g}  & 0 \ar[d]^{\tau_0}  \\
\mathrm{Ker}f    \ar[r]^{i_f} & Y \ar[r]^{f} & Z }

\begin{rem}
 自然性はエクササイズでお願いします。燃え尽きました。
\end{rem}


\end{document}
