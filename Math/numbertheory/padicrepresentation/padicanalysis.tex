%===============
%一行目に必ず必要
%文章の形式を定義
%===============
\documentclass{ujarticle}
%===============
%パッケージの定義、必要か不明
%===============
%この下4つを加えることで、mathbbが機能した
\usepackage{amsthm}
\usepackage{amsmath}
\usepackage{amssymb}
\usepackage{amsfonts}
%可換図式用パッケージ
\usepackage{amscd}
\usepackage[all]{xy}
\usepackage{tikz-cd}
%リンク用パッケージ
\usepackage[dvipdfmx]{hyperref}
%複数行コメント
%\usepackage{comment}

%タイトルデータ
\title{p進解析入門}
\author{ari}
\date{2016/9/22}


%===============
%定理環境の設定
%セクション毎
%===============
\newtheorem{thm}{Theorem}[section]
\newtheorem{dfn}[thm]{Definition}
\newtheorem{prop}[thm]{Propostion}
\newtheorem{lem}[thm]{Lemma}
\newtheorem{cor}[thm]{Corllary}
\newtheorem{epl}[thm]{Example}
\newtheorem*{prob}{Problem}
\newtheorem*{rem}{Remark}
\newtheorem{prf}{Proof}

\begin{document}

% タイトルを出力
\maketitle
% 目次の表示
\tableofcontents


\section{Introduction}
\label{sec:Introduction}

$p$進解析入門.$p$進解析について自分の復習も兼ねて,記載する.

\begin{itemize}
  \item $p$-adic位相の基本的な定義
  \item $\mathbb{Q}_p$の間の連続写像
  \item $p$-adic Banah空間の一般論?
  \item $\mathbb{C}_p$のべき級数における一般論
\end{itemize}

このあたりを調べて,最終的には以下の4つを理解し直したい.
\begin{itemize}
  \item $\mathbb{Z}_p$の間の連続写像の決定(Mahlerの定理)
  \item 全不連結性より,微分が消えるがconstでない写像の例
  \item $p$-adic L関数の定義と一意性
  \item Weil Conjecture
\end{itemize}

\section{$p$-adic topology}
\label{sec:$p$-adic topology}

\subsection{$\mathbb{Q}$の完備化}
\label{sub:\mathbb{Q}の完備化}

***********
ここは後で埋める.
***********

\subsection{$\mathbb{Q}_p$の位相的性質}
\label{sub:$qb}


$\mathbb{Q}_p$の位相を定義し、収束やコンパクト性、連結性などの位相空間の基本的な性質を調べる.

\begin{dfn}
  $\mathbb{Q}_p$は以下により距離空間となる.
  \begin{equation*}
    d : \mathbb{Q}_p \times \mathbb{Q}_p \to \mathbb{R}_{\ge 0},d(x,y)=p^{-v_p(x-y)}
  \end{equation*}
  これによって定める位相を$p$-adic位相という.
\end{dfn}

\begin{lem}
  上の$d$は距離の公理を満たす.確認は容易なので,各自の演習問題とする.
\end{lem}

\begin{prop}
 距離空間$(\mathbb{Q}_p,d)$は完備である.
\end{prop}
\begin{proof}
どうやって計算するんだっけ?
\end{proof}

\end{document}
