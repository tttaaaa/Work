%===============
%一行目に必ず必要
%文章の形式を定義
%===============
\documentclass{ujarticle}
%===============
%パッケージの定義、必要か不明
%===============
%この下4つを加えることで、mathbbが機能した
\usepackage{amsthm}
\usepackage{amsmath}
\usepackage{amssymb}
\usepackage{amsfonts}
%可換図式用パッケージ
\usepackage{amscd}
\usepackage[all]{xy}
\usepackage{tikz-cd}
%リンク用パッケージ
\usepackage[dvipdfmx]{hyperref}
%複数行コメント
%\usepackage{comment}

%タイトルデータ
\title{位相空間とザリスキ位相}
\date{2016/9/26}


%===============
%定理環境の設定
%セクション毎
%===============
\newtheorem{thm}{Theorem}[section]
\newtheorem{dfn}[thm]{Definition}
\newtheorem{prop}[thm]{Propostion}
\newtheorem{lem}[thm]{Lemma}
\newtheorem{cor}[thm]{Corllary}
\newtheorem{epl}[thm]{Example}
\newtheorem*{prob}{Problem}
\newtheorem*{rem}{Remark}
\newtheorem{prf}{Proof}

\begin{document}

% タイトルを出力
\maketitle
% 目次の表示
%\tableofcontents

\section{Introdcution}
\label{sec:Introdcution}
位相,特にZariski位相について説明する.
位相は高校数学から純粋数学への橋渡しになる革新的なアイディアであり,数学好きの方々には是非理解してもらいたいものである.
その重要性から数学を学んだ人にとって空気のようにありふれた存在でありつつも,その革新さ故に,多くの初学者を苦しめる概念である.
今回は9/24日に開催された数学カフェや楕円曲線を学ぶ上で出て来る"Zariski位相"に焦点を当てつつ,位相に親しみを持つことを目標に話す.
\subsection{構造主義}
\label{sub:構造主義}
導入として,構造主義について触れておく.なお,構造主義は本来哲学の概念であり,その概念に私は詳しくないため,誤っている可能性があることに注意せよ.数学では,一つの対象を調べる時,集合に構造が付いたものだと考えるとわかりやすいことがある..
例えば整数環$\mathbb{Z}$を考えてみよう.
これは集合としては$\{ \cdots, 0,1,2, \cdots \}$である.
よく知られるように,この集合には,加法や乗法といった演算が定義でき,"環"としての構造を持っている.つまり,$\mathbb{Z}$は集合に加法や乗法といった付加構造がついたものと考えることができる.
こうして数学の対象を集合とその上の構造とみることで,より理解を促進させる考え方を構造主義という.
構造主義はBourbakiを中心に1960年頃に発展した考え方である.集合論中心で,圏論を無視している部分もあり,現在は数学の中心的なものの見方となっていない.しかし,現代数学の基礎づけをした重要な考え方である.
\begin{rem}
  付加構造も含め対象を考えると対象自体には一見変化がないが,対象同士の関係が異なる.
  例えば,$\mathbb{Z}$と$\mathbb{Q}$は集合としては同じ(同型)であるが,環としては同じ(同型)でない.つまり付加構造により,"同じ"とされるものが異なる.逆の言い方をすると,"異なる"とされるものが変わる.付加構造として重要な構造等は以下などがある.
  \begin{enumerate}
    \setlength{\parskip}{0cm} % 段落間
    \setlength{\itemsep}{0cm} % 項目間
    \item 代数構造
    \item 位相構造
    \item 微分構造
    \item 順序構造
  \end{enumerate}
\end{rem}

\section{位相空間}
\label{sec:位相空間}
\subsection{位相でできること}
\label{sub:位相でできること}
位相の定義を与える前に,位相を用いることで何ができるかを説明する.
位相によって議論したいものは以下の2つである.
\begin{itemize}
  \item 極限(収束)
  \item 連続
\end{itemize}

まず極限について考える.極限は高校では以下のように記述される.
ある実数係数の点列$\{ x_n \}$がある値$x$に十分"近い"時,
\begin{equation*}
 \lim_{n \to \infty} x_n=x
\end{equation*}
と書く.
近いとは,どういうことだろうか,直感的にはイメージできるが,それを数学的に定式化するできるだろうか.
実数の場合は,任意の$\epsilon \gt 0$に対し,ある$n$が存在し,$n$以上となる全ての$m$に対して,$|x - x_m| \le \epsilon$となることである.つまり,極限は点列に対し,点列を最もよく近似した点ということである.しかし,極限は全ての点列に対して,必ず存在するとは限らない.実数においても,そのことは簡単に確かめられる.

極限は集合内の話であったが,それを集合と集合の間に拡張したものが連続である.つまり,$f:X \to Y$に対し,$X$で十分近いものが$Y$で十分近いものへ映る時,連続という.高校では$\{x_n \}$が$x$に収束する時,以下が成り立つと,
\begin{equation*}
 \lim_{x_n \to x}f(x_n)=f(x)
\end{equation*}
$f$が連続であると定義されていた.

こうした概念をもっと広い集合に対して,定義するのが位相である.

\subsection{位相の定義}
\label{sub:位相の定義}
最初に位相を定義し,それが"近さ"を定義していることをみよう.
(TBD)

\begin{dfn}[開集合の公理]
 集合$X$を空でない集合とする.集合$X$のべき集合$\mathfrak{P}(X)$に対し,その部分集合$O_X$が以下を満たす時,$O_X$は$X$に位相構造を定めるという.
 \begin{enumerate}
   \item $\emptyset , X \in O_X$.
   \item $O_1,O_2 \in O_X$に対し,$O_1 \cap O_2 \in O_X$.
   \item 集合$\Lambda$の元$\lambda$に対し,$O_{\lambda} \in O_X$とする.この時,$ \bigcup_{\lambda \in \Lambda}O_{\lambda} \in O_X$.
 \end{enumerate}.
 $O_X$の元を開集合という.
\end{dfn}

\begin{dfn}[閉集合の公理]
 集合$X$を空でない集合とする.集合$X$のべき集合$\mathfrak{P}(X)$に対し,その部分集合$V_X$が以下を満たす時,$V_X$は$X$に位相構造を定めるという.
 \begin{enumerate}
   \item $\emptyset , X \in V_X$.
   \item $V_1,V_2 \in V_X$に対し,$V_1 \cup V_2 \in V_X$.
   \item 集合$\Lambda$の元$\lambda$に対し,$V_{\lambda} \in V_X$とする.この時,$\displaystyle \cap_{\lambda \in \Lambda}V_{\lambda} \in V_X$.
 \end{enumerate}
 $V_X$の元を閉集合という.
\end{dfn}

\begin{rem}
位相には開集合による定義,閉集合による定義,近傍系による定義があるが,どれも同値なので,好きなものを採用せよ.
\end{rem}
\begin{rem}
 本当にいらない余談であるが,私は位相空間が最初,わからなくて大変苦労したが,多様体論などで使っているうちに,そんなものかと消化した.そのため位相空間がわからない場合は使いながら理解するのが一番よいかもしれない.
\end{rem}

$x$と$y$が"近い"は"多く"の$x$を含む開集合から$y$も含むこと.

\begin{dfn}
 位相空間$X,Y$に対し,$X \to Y$が連続とは,任意の$Y$の開集合$O$に対し,$f^{-1}(O)$が開集合となること.(閉集合でも同様)
\end{dfn}
この定義のメリットを説明する.
実数のときよりも定義がシンプルである.
また,元を取らずに定義できているため,大域的にわかりやすい.

しかし,これではすぐにはわからない部分もあるため,連続を少し言い換えてみる.
\begin{dfn}
 $f:X \to Y$と,$X$の元$x$に対し,$f(x) \in Y$を含む任意の開集合$O'$をとる.$x$を含む開集合$O$で$f(O) \subset O'$となるとき,$f$は連続という.
\end{dfn}
この連続の定義は上の連続の定義と同値になる.


\subsection{例}
\label{sub:例}

射影空間(コンパクトな空間の例)

原点が2点ある空間(極限が2点以上になる.)
(TBD)
\subsection{コンパクト,ハウスドルフ}
\label{sub:コンパクト,ハウスドルフ}
ハウスルドルフは,分離的,つまり,点と点は位相含め異なっているということである.特に,一点集合は必ず閉集合になる.ハウスドルフでない世界では,点列の極限が複数あるという異常な事態が発生する.
コンパクトは,有界かつ閉(完備)という意味である.
実数の場合は閉区間$[-1,2]$等がコンパクトである.
数学においては非常に重要な概念だが,今回の話には関係がないため
,話は触れない.
ただし,位相空間の基本的な定理である以下をRemarkとして記載していおく.
\begin{thm}
コンパクト位相空間$X$からハウスドルフ位相空間$Y$からの連続な全単射は同相写像になる.
\end{thm}





\section{ザリスキ位相}
\label{sub:ザリスキ位相}

$K^n$に閉集合を用いて,ザリスキ位相を定義する.
\begin{dfn}
 $V \subset K^n$がある多項式環のイデアル$I \subset K[X_1,\dots,X_n]$を用いて,
\begin{equation*}
   V = \{ (x_1,\dots,x_n) \in K^n | \mbox{任意の} f \in I \mbox{に対し} f(x_1, \dots,x_n)=0 \}
\end{equation*}
と書ける時,$V$を閉集合と定める.またこの$V$を$V(I)$と書く.
\end{dfn}

$V(I)$が閉集合の公理を満たしていることを確認しよう.
\begin{prop}
 以下が成り立つ.
 \begin{enumerate}
   \item $V(I) = \emptyset , V(I) = K^n$となるイデアル$I$が存在する.
   \item $V(I) \cup V(J) = V(IJ)$
   \item $\bigcap_{\lambda \in \Lambda}V(I_{\lambda})=V(\sum_{\lambda \in \Lambda}V(I_\lambda))$
 \end{enumerate}
\end{prop}
証明は読者の演習問題とする.

この位相をザリスキ位相という.
$K$として$\mathbb{C}$を取る.$\mathbb{C}$には自然にユークリッド位相が定義されているが,ザリスキ位相はユークリッド位相とは異なっている.$V(I)$は必ず$\mathbb{C}^n$の閉集合になっているが,逆は成り立たない.
例えば,$n=1$とすると,$K[X]$の任意のイデアルは一元で生成される.
1変数多項式は解を高々有限個しか持たないため,
$V(I)$は$\mathbb{C},\emptyset,$有限集合のどれかになる.
つまり閉区間$[a,b]$が閉集合にならない.
代数幾何で自然に出てくる位相であるが,この位相はユークリッド位相に比べ,真に小さい位相となっている.


ザリスキ位相を少し異なる点から解釈しよう.
\begin{thm}
 $\mathbb{C}$(代数閉体でよい)上の多項式環$\mathbb{C}[X_1,\dots,X_n]$の極大イデアルは$(X_1 -a_1, \dots ,X_n - a_n)$と書ける.
\end{thm}

ヒルベルトの零点定理より,一点と極大イデアルが一対一に対応する.
この対応により,極大イデアルのなす集合に位相を定めることができる.
つまり,極大イデアル全体のなす集合を空間だと思えるのである!!!
これは大きな発想の転換である.代数幾何においては,まず,空間があって,その空間上の関数を考えるのではなく,関数(とみなせるもの)があれば,それが自然に空間を定めていると考えることができる.

これを徹底的におしすすめた結果,
関数とは環であり,その関数を与える空間は素イデアル全体のなす集合であると考えるとうまくいくことがわかった.この考えを推し進めて,生まれたものが,(アフィン)スキームである.
環$R$に対し,ザリスキ位相をどう定めるのか見ておこう.
\begin{dfn}
 $V \subset \mathrm{Spec}R$があるイデアル$I \subset R$を用いて,
\begin{equation*}
   V = \{ \mathfrak{p} \in \mathrm{Spec}R |  I \subset \mathfrak{p} \}
\end{equation*}
と書ける時,$V$を閉集合と定める.またこの$V$を$V(I)$と書く.
\end{dfn}
これは確かに閉集合を定めることが(計算すると)わかる.

この位相では,少し変なことが起きる.
ユークリッド位相などでは一点は必ず閉集合となっていた.
これはハウスドルフな位相空間においては必ず成り立つ.
しかし,この空間では,一点は閉集合になるとは限らない.
例えば,$\mathrm{Spec}\mathbb{Z}$を考える.
$(0)$は素イデアルなので,$\mathrm{Spec}\mathbb{Z}$の元となる,
$(0) \in V(I)$はザリスキ位相の定義から,$I \subset (0)$を意味するが,
これは$I = (0)$を意味する.任意の素イデアル$\mathfrak{p}$は$0$を要素に含むので,$V((0))=\mathrm{Spec}\mathbb{Z}$となる.
このため$(0)$は閉集合でないこともわかった.

この位相も代数幾何で使いなれてくると自然に感じる.
しかし,ヴェイユ予想等で出てくる問題を考えるコホモロジー(位相空間上の層を定義し,その層のなす圏の導来関手)がよい性質を持つにはまだ位相が足りない(らしい).
そのため,グロタンディークは位相空間ではないものを,位相とみなしグロタンディークトポロジーを定めた.これをスキーム上のエタール被覆に対し適用して,定めたコホモロジーがエタールコホモロジーである.


蛇足であるが,環準同型$f:R_1 \to R_2$に対し,
\begin{equation*}
  \phi:\mathrm{Spec}R_2 \ to \mathrm{Spec}R_1, \mathfrak{p}_2 \mapsto f^{-1}(\mathfrak{p})
\end{equation*}が定義され,これが連続になっている.


\end{document}
