%===============
%一行目に必ず必要
%文章の形式を定義
%===============
\documentclass{ujarticle}
%===============
%パッケージの定義、必要か不明
%===============
%この下4つを加えることで、mathbbが機能した
\usepackage{amsthm}
\usepackage{amsmath}
\usepackage{amssymb}
\usepackage{amsfonts}
%可換図式用パッケージ
\usepackage{amscd}
\usepackage[all]{xy}
\usepackage{tikz-cd}
%リンク用パッケージ
\usepackage[dvipdfmx]{hyperref}
%複数行コメント
%\usepackage{comment}

%タイトルデータ
\title{Modular Formとその周辺}
\author{take}
\date{2016/August}


%===============
%定理環境の設定
%セクション毎
%===============
\newtheorem{thm}{Theorem}[section]
\newtheorem{dfn}[thm]{Definition}
\newtheorem{prop}[thm]{Propostion}
\newtheorem{lem}[thm]{Lemma}
\newtheorem{ex}[thm]{Example}
\newtheorem*{prob}{Problem}
\newtheorem*{rem}{Remark}
\newtheorem{prf}{Proof}

\begin{document}

% タイトルを出力
\maketitle
% 目次の表示
\tableofcontents


\section{Introduction}
\label{sec:Introduction}

保型形式について概説する。
今回の説明は以下を理解することを目的に行う。
\begin{itemize}
  \setlength{\parskip}{0cm} % 段落間
  \setlength{\itemsep}{0cm} % 項目間


  \item 保型形式の定義
  \item 保型形式を考える背景
  \item 数学における興味深い対象
  \item 数学における自然な疑問
  \item 最低限の理論の厳密さ
\end{itemize}

本サーベイにおいては、保型形式は解析的に定義されるため、まず、解析的な内容の準備を用意した。
これらの準備を適宜みならがら、保型形式の定義から基本領域、フーリエ級数展開とL関数等、代数を使わずとも定義できる
内容を論じる。その後、保型形式全体のなす空間がベクトル空間となることを確認し、内積、固有値、次元などについて議論する。
以降は概略のみであるが、保型形式の広がりの深さをみせるため、
保型形式と楕円関数、楕円曲線との関係、また保型形式のL関数と代数体のL関数などの関係にふれる。
最後にフェルマーの最終定理とつながる、Eichler-Shimura理論について触れる。
なお、今回は、Langlands Problemや岩澤理論は高度すぎると判断し、特に触れなかった。
詳しいことは参考文献を参考にせよ。

後半は詳しいことを私も理解していないため、誤りがある可能性が高いことを最初に触れておく。
もし、誤りが発見された場合は通知願う。



\begin{rem}
  一般的に解析的、代数的、幾何的に定まった意味はない。特に理論が進歩していくと、何が代数的で、何が幾何的で、
  何が解析的なのかがよくわからなくなることがある。しかし、ここでは、そのような特殊な状況は考慮せず、以下の意味で使い分ける。
 \begin{description}
    \item[代数的定義] 群、環、体、ベクトル空間などの代数的構造によって定まる定義
    \item[幾何的定義] 多様体、ホモロジー等、空間や空間の不変量により定まる定義
    \item[解析的定義] 関数や微積分などにより定まる定義
 \end{description}
\end{rem}

\section{Basic Knowlede of this survey}
\label{sec:Basic Knowlede of this survey}

\begin{thm}[フーリエ展開の収束]
\label{thm:fourier}
 連続関数$f:\mathbb{R} \to \mathbb{C}$が$f(x + 1) = f(x)$を満たすとする。
 すると、$f= \displaystyle  \sum_{n= - \infty }^{\infty}a_ne^{2 \pi i x n}$となる。\\
 ただし、$a_n e^{2 \pi i x n}= \int_{-\frac{1}{2}}^{\frac{1}{2}}f(x)e^{-2 \pi i x n}dx$
また、この等式が成り立つとき、$f$はフーリエ展開可能という。
\end{thm}

\section{Definition of Modular Form and Basic Propety}
\label{sec:Modular Form}

\subsection{$SL(2.\mathbb{Z})$ and congruenc gruop}
\label{sub:$SL(2.Z and congurenc gruop}


\subsection{Definition of modular form}
\label{sub:Definition of modular form}
保型形式は名前の通り、なんらかの形を保つものである。
形を保つというのは、数学ではよく、群の作用で不変という形で定義される。
群の作用が何かをここでは定義しないが、"何か"を"かけた"場合に元とほとんど同じということを指すと思ってもらいたい。

\noindent では実際に保型形式をみていこう。まず、"かけられる対象"が何で、"かける対象"が何かを定義する。

\noindent \bf{かける対象:}
\begin{equation*}
  \Gamma(N) := \{
  \begin{pmatrix}
    a & b \\
    c & d \\
  \end{pmatrix}
  \in \mathrm{SL}_2(\mathbb{Z}) | a \equiv 1,b \equiv 0,c \equiv 0, d \equiv 1 \mathrm{mod} N
  \}
\end{equation*}
\begin{center}
  $N$は1以上の整数を指す。$N$が1の場合、$\Gamma(N) = \mathrm{SL_2(\mathbb{Z})}$となる。
\end{center}

\noindent \bf{かけられる対象:}
$\mathcal{H}$から$\mathbb{C}$への有理型関数$f$全体

\begin{rem}
  かける対象、かけられる対象は実際はそれらの元と書くべきかもしれない。ただ、数学で対象を考えるというとき、
  少なくとも自分は、対象を性質や条件で決めることが多い。そのため一つの特定の元ではなく、全体をさした方が自然に感じる。
\end{rem}

\noindent かける対象とかけられる対象が決まったので、"かける"を以下で定義する。

\begin{eqnarray*}
  \cdot :  \Gamma(N) \times \mathrm{Hom}_{Rat}\{\mathcal{H},\mathcal{H}\} & \to & \mathrm{Hom}_{Rat}\{\mathcal{H},\mathcal{H}\} \\
  (\gamma,f) &\mapsto &\gamma \cdot f(z) = f(\frac{az + b}{cz + d}) \\
\mbox{ただし、} \gamma =
\begin{pmatrix}
  a & b \\
  c & d \\
\end{pmatrix}
\end{eqnarray*}
かけることまで定義できたので、保型形式を定義しよう。
以降では保型形式はすべて英語のmodular form,cusp formなどの用語を使う。
modular form,automorphic formがともに保型形式と訳されていること、
cusp formを日本語で書いているものをほとんどとみたことがないためである。

\begin{dfn}
  有理型関数$f:\mathcal{H} \to \mathbb{C}$がunrestrictedなweight $k$のmodular formとは、以下が成り立つことをいう。 \\
  $\forall  \gamma \in \Gamma(N),  \gamma \cdot f(z) = (cz + d)^{k}f(z)$
\end{dfn}
\begin{rem}
  定義を見たときは、まずこの定義が意味することを考えたい。
  真剣に数学書を読むなら、定義に対する疑問3個、定義が成り立つ例1個、定義が成り立たない例1個ぐらいは考えたい。
\end{rem}
この定義をみて、私が気になることをいくつかあげる。
\begin{itemize}
  \setlength{\parskip}{0cm} % 段落間
  \setlength{\itemsep}{0cm} % 項目間
  \item $a,b$によらず$c,d$によるのは不自然に感じる。なぜ、そのようなものを考えるのか。
  ⇒楕円関数など自然な例が多数あるため
  \item $\mathcal{H}$を割った空間の関数として考えられないのか。
  ⇒簡単。
  \item $\gamma \cdot f (z) = f(z)$となる$f$はあるのか。
  ⇒ある。
\end{itemize}

\subsection{fourier expansion of modular form}
\label{sub:fourier expansion of modular form}
上で定義したunrestricted modular formがフーリエ展開可能であることを示そう。
以下の方針で示す。
\begin{enumerate}
  \setlength{\parskip}{0cm} % 段落間
  \setlength{\itemsep}{0cm} % 項目間
  \item 周期的であること。
  \item ある点の近傍でフーリエ展開できること
  \item 任意の点でフーリエ展開で近似できること
\end{enumerate}
\begin{proof}[\bf{周期的であること}]

$\gamma = \begin{pmatrix}
  1 & 1 \\
  0 & 1 \\
\end{pmatrix} \in \mathrm{SL}_2(\mathbb{Z})$
をとる。この時、$\gamma \cdot z = z +1, cz+d =1$となるため、
unrestricted modular form$f$に対し、$\gamma \cdot f(z)= f(z +1)=f(z)$となる。
よって、実軸にそって周期関数となることがわかる。
\end{proof}
\begin{proof}{\bf{ある点の近傍でフーリエ展開できること}}
周期的であることはわかったが、上の関数は有理型であるため、特異点が存在する場合がある。
しかし、ある$y \in \mathbb{R}$をとると、$f(x +iy) \neq \infty$となることを示す。
それが言えれば、その$y$に対し、\ref{thm:fourier}を展開することで、ある点の近傍でフーリエ展開できることが示せる。
実際に、示そう。もし$ \forall y \in \mathbb{R}$に対し、ある$x$が存在し、$f(x +iy)=\infty$となったとする。
適当な$r \in \mathbb{R_{\ge 0}}$をとり、$[x,x+1] \times [y,y +r]$を考える。
$f(z + 1)=f(z)$より、$[x,x+1] \times [y,y +r]$上に無限個の特異点が存在する。
これはコンパクト性から集積点が存在することになる。
これは、有理型関数の特異点が真性特異点でないことに矛盾する。
よって、ある点の近傍でフーリエ展開できることがわかった。
\end{proof}
\begin{proof}{\bf{任意の点でフーリエ展開で近似できること}}
  任意の点$z \in \mathcal{H}$に対し、十分小さい近傍Uをとると、$z' \in U s.t \mathrm{Im}z' \neq \mathrm{Im}z$
  で、フーリエ展開できる。これはフーリエ展開できないとすると、近傍(を含むあるコンパクト集合)に無限個の特異点が存在することになり、わかる。
  また、$z$を除き正則であるように取れる。
  $z$が特異点でない場合は$z$の近傍で正則となり、わからん。

\end{proof}

ただし、特異点があることに注意せよ。
今、$f(z)$は有理型関数のため、真性特異点が存在しないことに注意すると、

\ref{thm:fourier}


\begin{rem}
  筆者は解析全般に詳しくないため、ここまで議論した。確認したすべてのmdfmの教科書でここの証明がなかったので、
  わかっている人からすると明らかなこと、あるいは明らかにできる定理があるのかもしれない。
\end{rem}

unrestricted modular formの定義
modular form/cusp formの定義
tu


\begin{thm}{[k],[11.74 Eichler-Shimura]} \\
  $f(\tau)=\sum_{n=1}^{\infty}c_ne^{2 \pi i n \tau}$を$S_2(\Gamma_0(N))$のnew formであって、
  $c_1=1$となるものとする。 $c_n \in \mathbb{Z}$とすると、以下を満たすpair $(E, \nu)$が存在する
  \begin{enumerate}
    \item $E$ は $\mathbb{Q}$上の楕円曲線であり、$(E, \nu)$は$\mathbb{Q}$上のアーベル多様体$J$を部分多様体で割ったものとなる。
  \end{enumerate}

\end{thm}


\begin{thebibliography}{数字}
  \bibitem{K} Elliptic Curves.・Anthony W. Knapp
  \bibitem{M} Modular Functions and Modular Forms・J. S. Milne
  \bibitem{キーN} 参考文献の名前・著者N
\end{thebibliography}

\end{document}
