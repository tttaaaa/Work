%===============
%一行目に必ず必要
%文章の形式を定義
%===============
\documentclass{ujarticle}
%===============
%パッケージの定義、必要か不明
%===============
%この下4つを加えることで、mathbbが機能した
\usepackage{amsthm}
\usepackage{amsmath}
\usepackage{amssymb}
\usepackage{amsfonts}
%可換図式用パッケージ
\usepackage{amscd}
\usepackage[all]{xy}
\usepackage{tikz-cd}
%リンク用パッケージ
\usepackage[dvipdfmx]{hyperref}
%複数行コメント
%\usepackage{comment}

%タイトルデータ
\title{Modular Formとその周辺}
\author{ta}
\date{2016/August}


%===============
%定理環境の設定
%セクション毎
%===============
\newtheorem{thm}{Theorem}[section]
\newtheorem{dfn}[thm]{Definition}
\newtheorem{prop}[thm]{Propostion}
\newtheorem{lem}[thm]{Lemma}
\newtheorem{ex}[thm]{Example}
\newtheorem*{prob}{Problem}
\newtheorem*{rem}{Remark}
\newtheorem{prf}{Proof}

\begin{document}

% タイトルを出力
\maketitle
\section{Introduction}
\label{sec:Introduction}

保型形式について概説する。
今回の説明は以下を理解することを目的に行う。
\begin{itemize}
  \setlength{\parskip}{0cm} % 段落間
  \setlength{\itemsep}{0cm} % 項目間


  \item 保型形式の定義
  \item 保型形式を考える背景
  \item 数学における興味深い対象
  \item 数学における自然な疑問
  \item 最低限の理論の厳密さ
\end{itemize}

本サーベイにおいては、保型形式は解析的に定義されるため、まず、解析的な内容の準備を用意した。
これらの準備を適宜みならがら、保型形式の定義から基本領域、フーリエ級数展開とL関数等、代数を使わずとも定義できる
内容を論じる。その後、保型形式全体のなす空間がベクトル空間となることを確認し、内積、固有値、次元などについて議論する。
以降は概略のみであるが、保型形式の広がりの深さをみせるため、
保型形式と楕円関数、楕円曲線との関係、また保型形式のL関数と代数体のL関数などの関係にふれる。
最後にフェルマーの最終定理とつながる、Eichler-Shimura理論について触れる。
なお、今回は、Langlands Problemや岩澤理論は高度すぎると判断し、特に触れなかった。
詳しいことは参考文献を参考にせよ。

後半は詳しいことを私も理解していないため、誤りがある可能性が高いことを最初に触れておく。
もし、誤りが発見された場合は通知願う。



\begin{rem}
  一般的に解析的、代数的、幾何的に定まった意味はない。特に理論が進歩していくと、何が代数的で、何が幾何的で、
  何が解析的なのかがよくわからなくなることがある。しかし、ここでは、そのような特殊な状況は考慮せず、以下の意味で使い分ける。
 \begin{description}
    \item[代数的定義] 群、環、体、ベクトル空間などの代数的構造によって定まる定義
    \item[幾何的定義] 多様体、ホモロジー等、空間や空間の不変量により定まる定義
    \item[解析的定義] 関数や微積分などにより定まる定義
 \end{description}
\end{rem}

\section{Basic Knowlede of this survey}
\label{sec:Basic Knowlede of this survey}


\section{Definition of Modular Form and Basic Propety}
\label{sec:Modular Form}

\subsection{$SL(2.\mathbb{Z})$ and congurenc gruop}
\label{sub:$SL(2.Z and congurenc gruop}


\subsection{definition of modular form}
\label{sub:definition of modular form}
保型形式は名前の通り、なんらかの形を保つものである。
形を保つというのは、数学ではよく、群の作用で不変という形で定義される。
群の作用が何かをここでは定義しないが、"何か"を"かけた"場合に元とほとんど同じということを指すと思ってもらいたい。

\noindent では実際に保型形式をみていこう。まず、"かけられる対象"が何で、"かける対象"が何かを定義する。

\noindent \bf{かける対象:}
\begin{equation*}
  \Gamma(N) := \{
  \begin{pmatrix}
    a & b \\
    c & d \\
  \end{pmatrix}
  \in \mathrm{SL}_2(\mathbb{Z}) | a \equiv 1,b \equiv 0,c \equiv 0, d \equiv 1 \mathrm{mod} N
  \}
\end{equation*}
\begin{center}
  $N$は1以上の整数を指す。$N$が1の場合、$\Gamma(N) = \mathrm{SL_2(\mathbb{Z})}$となる。
\end{center}

\noindent \bf{かけられる対象:}
$\mathcal{H}$から$\mathcal{H}$への解析的関数$f$全体
(もしかしたら有理型でいいかもしれない。)
\begin{rem}
  かける対象、かけられる対象は実際はそれらの元と書くべきかもしれない。ただ、数学で対象を考えるというとき、
  少なくとも自分は、対象を性質や条件で決めることが多い。そのため一つの特定の元ではなく、全体をさした方が自然に感じる。
\end{rem}

\noindent かける対象とかけられる対象が決まったので、"かける"を以下で定義する。

\begin{eqnarray*}
  \cdot :  \Gamma(N) \times \mathrm{Hom}_{Rat}\{\mathcal{H},\mathcal{H}\} & \to & \mathrm{Hom}_{Rat}\{\mathcal{H},\mathcal{H}\} \\
  (\gamma,f) &\mapsto &\gamma \cdot f(z) = f(\frac{az + b}{cz + d}) \\
\mbox{ただし、} \gamma =
\begin{pmatrix}
  a & b \\
  c & d \\
\end{pmatrix}
\end{eqnarray*}
かけることまで定義できたので、保型形式を定義しよう。
以下では保型形式と言わずにmodular form,cusp formなどの用語を使う。

\begin{dfn}
  $f$がunrestricted weight $k$のmodular formとは、以下が成り立つことをいう。 \\
  $\forall  \gamma \in \Gamma(N),  \gamma \cdot f(z) = (cz + d)^{k}f(z)$
\end{dfn}
\begin{rem}
  定義を見たときは、まずこの定義が意味することを考えたい。
  真剣に数学書を読むなら、定義に対する疑問3個、定義が成り立つ例2個、定義が成り立たない例1個ぐらいは考えたい。
\end{rem}
この定義をみて、私が気になることをいくつかあげる。
\begin{itemize}
  \setlength{\parskip}{0cm} % 段落間
  \setlength{\itemsep}{0cm} % 項目間
  \item $a,b$によらず$c,d$によるのは不自然に感じる。なぜ、そのようなものを考えるのか。
  \item $f$は有理型関数に限っているが、他の場合はどうか。
  \item $\mathrm{H}$を割った空間の関数として考えられないのか。
  \item $\gamma \cdot f (z) = f(z)$となる$f$はあるのか。
\end{itemize}


\begin{thm}{[k],[11.74 Eichler-Shimura]} \\
  $f(\tau)=\sum_{n=1}^{\infty}c_ne^{2 \pi i n \tau}$を$S_2(\Gamma_0(N))$のnew formであって、
  $c_1=1$となるものとする。 $c_n \in \mathbb{Z}$とすると、以下を満たすpair $(E, \nu)$が存在する
  \begin{enumerate}
    \item $E$ は $\mathbb{Q}$上の楕円曲線であり、$(E, \nu)$は$\mathbb{Q}$上のアーベル多様体$J$を部分多様体で割ったものとなる。
  \end{enumerate}

\end{thm}


\begin{thebibliography}{数字}
  \bibitem{K} Elliptic Curves.・Anthony W. Knapp
  \bibitem{M} Modular Functions and Modular Forms・J. S. Milne
  \bibitem{キーN} 参考文献の名前・著者N
\end{thebibliography}

\end{document}
