%===============
%一行目に必ず必要
%文章の形式を定義
%===============
\documentclass{ujarticle}
%===============
%パッケージの定義、必要か不明
%===============
%この下4つを加えることで、mathbbが機能した
\usepackage{amsthm}
\usepackage{amsmath}
\usepackage{amssymb}
\usepackage{amsfonts}
%可換図式用パッケージ
\usepackage{amscd}
\usepackage[all]{xy}
\usepackage{tikz-cd}
%リンク用パッケージ
\usepackage[dvipdfmx]{hyperref}
%複数行コメント
%\usepackage{comment}

%タイトルデータ
\title{Basic Theory of Modular Form}
\author{take}
\date{2016/August}


%===============
%定理環境の設定
%セクション毎
%===============
\newtheorem{thm}{Theorem}[section]
\newtheorem{dfn}[thm]{Definition}
\newtheorem{prop}[thm]{Propostion}
\newtheorem{lem}[thm]{Lemma}
\newtheorem{ex}[thm]{Example}
\newtheorem*{prob}{Problem}
\newtheorem*{rem}{Remark}
\newtheorem{prf}{Proof}

\begin{document}

% タイトルを出力
\maketitle
% 目次の表示
\tableofcontents


\section{Introduction}
\label{sec:Introduction}

保型形式について概説する。
今回の説明は以下を理解することを目的に行う。
\begin{itemize}
  \setlength{\parskip}{0cm} % 段落間
  \setlength{\itemsep}{0cm} % 項目間


  \item 保型形式の定義
  \item 保型形式を考える背景
  \item 数学における興味深い対象
  \item 数学における自然な疑問
  \item 最低限の理論の厳密さ
\end{itemize}

本サーベイにおいては、保型形式は解析的に定義されるため、まず、解析的な内容の準備を用意した。
これらの準備を適宜みならがら、保型形式の定義から基本領域、フーリエ級数展開とL関数等、代数を使わずとも定義できる
内容を論じる。その後、保型形式全体のなす空間がベクトル空間となることを確認し、内積、固有値、次元などについて議論する。
以降は概略のみであるが、保型形式の広がりの深さをみせるため、
保型形式と楕円関数、楕円曲線との関係、また保型形式のL関数と代数体のL関数などの関係にふれる。
最後にフェルマーの最終定理とつながる、Eichler-Shimura理論について触れる。
なお、今回は、Langlands Problemや岩澤理論は高度すぎると判断し、特に触れなかった。
詳しいことは参考文献を参考にせよ。

後半は詳しいことを私も理解していないため、誤りがある可能性が高いことを最初に触れておく。
もし、誤りが発見された場合は通知願う。



\begin{rem}
  一般的に解析的、代数的、幾何的に定まった意味はない。特に理論が進歩していくと、何が代数的で、何が幾何的で、
  何が解析的なのかがよくわからなくなることがある。しかし、ここでは、そのような特殊な状況は考慮せず、以下の意味で使い分ける。
 \begin{description}
    \item[代数的定義] 群、環、体、ベクトル空間などの代数的構造によって定まる定義
    \item[幾何的定義] 多様体、ホモロジー等、空間や空間の不変量により定まる定義
    \item[解析的定義] 関数や微積分などにより定まる定義
 \end{description}
\end{rem}

\section{Basic Knowlede of this survey}
\label{sec:Basic Knowlede of this survey}

\begin{thm}[フーリエ展開の収束]
\label{thm:fourier}
 連続関数$f:\mathbb{R} \to \mathbb{C}$が$f(x + 1) = f(x)$を満たすとする。
 すると、$f= \displaystyle  \sum_{n= - \infty }^{\infty}a_ne^{2 \pi i x n}$となる。\\
 ただし、$a_n e^{2 \pi i x n}= \int_{-\frac{1}{2}}^{\frac{1}{2}}f(x)e^{-2 \pi i x n}dx$
また、この等式が成り立つとき、$f$はフーリエ展開可能という。
\end{thm}

\section{Definition of Modular Form and Basic Propety}
\label{sec:Modular Form}

\subsection{$SL(2.\mathbb{Z})$ and congruenc gruop}
\label{sub:$SL(2.Z and congurenc gruop}


\subsection{Definition of modular form}
\label{sub:Definition of modular form}
\subsubsection{Definition of unrestricted modular from}
\label{subs:Definition of unrestricted modular from}

保型形式は名前の通り、なんらかの形を保つものである。
形を保つというのは、数学ではよく、群の作用で不変という形で定義される。
群の作用が何かをここでは定義しないが、"何か"を"かけた"場合に元とほとんど同じということを指すと思ってもらいたい。

\noindent では実際に保型形式をみていこう。まず、"かけられる対象"が何で、"かける対象"が何かを定義する。

\noindent \bf{かける対象:}
\begin{equation*}
  \Gamma(N) := \{
  \begin{pmatrix}
    a & b \\
    c & d \\
  \end{pmatrix}
  \in \mathrm{SL}_2(\mathbb{Z}) | a \equiv 1,b \equiv 0,c \equiv 0, d \equiv 1 \mathrm{mod} N
  \}
\end{equation*}
\begin{center}
  $N$は1以上の整数を指す。$N$が1の場合、$\Gamma(N) = \mathrm{SL_2(\mathbb{Z})}$となる。
\end{center}

\noindent \bf{かけられる対象:}
$\mathcal{H}$から$\mathbb{C}$への正則関数$f$全体

\begin{rem}
  かける対象、かけられる対象は実際はそれらの元と書くべきかもしれない。ただ、数学で対象を考えるというとき、
  少なくとも自分は、対象を性質や条件で決めることが多い。そのため一つの特定の元ではなく、全体をさした方が自然に感じる。
\end{rem}

\noindent かける対象とかけられる対象が決まったので、"かける"を以下で定義する。

\begin{eqnarray*}
  \cdot :  \Gamma(N) \times \mathrm{Hom}_{Rat}\{\mathcal{H},\mathcal{H}\} & \to & \mathrm{Hom}_{Rat}\{\mathcal{H},\mathcal{H}\} \\
  (\gamma,f) &\mapsto &\gamma \cdot f(z) = f(\frac{az + b}{cz + d}) \\
\mbox{ただし、} \gamma =
\begin{pmatrix}
  a & b \\
  c & d \\
\end{pmatrix}
\end{eqnarray*}
かけることまで定義できたので、保型形式を定義しよう。
以降では保型形式はすべて英語のmodular form,cusp formなどの用語を使う。
modular form,automorphic formがともに保型形式と訳されていること、
cusp formを日本語で書いているものをほとんどとみたことがないためである。

\begin{dfn}
  正則関数$f:\mathcal{H} \to \mathbb{C}$がunrestrictedなweight $k$のmodular formとは、以下が成り立つことをいう。 \\
  $\forall  \gamma \in \Gamma(N),  \gamma \cdot f(z) = (cz + d)^{k}f(z)$
\end{dfn}
\begin{rem}
  定義を見たときは、まずこの定義が意味することを考えたい。
  真剣に数学書を読むなら、定義に対する疑問3個、定義が成り立つ例1個、定義が成り立たない例1個ぐらいは考えたい。
\end{rem}
この定義をみて、私が気になることをいくつかあげる。
\begin{itemize}
  \setlength{\parskip}{0cm} % 段落間
  \setlength{\itemsep}{0cm} % 項目間
  \item $a,b$によらず$c,d$によるのは不自然に感じる。なぜ、そのようなものを考えるのか。
  ⇒楕円関数など自然な例が多数あるため
  \item $\mathcal{H}$を割った空間の関数として考えられないのか。
  ⇒簡単。
  \item $\gamma \cdot f (z) = f(z)$となる$f$はあるのか。
  ⇒ある。
\end{itemize}

\subsubsection{fourier expansion of modular form}
\label{sub:fourier expansion of modular form}
上で定義したunrestricted modular formがフーリエ展開可能であることを示そう。
以下の方針で示す。
\begin{enumerate}
  \setlength{\parskip}{0cm} % 段落間
  \setlength{\itemsep}{0cm} % 項目間
  \item x方向に周期的であること。
  \item フーリエ係数が$y$方向に不変なこと
\end{enumerate}
\begin{proof}[\bf{周期的であること}]


$\gamma = \begin{pmatrix}
  1 & 1 \\
  0 & 1 \\
\end{pmatrix} \in \mathrm{SL}_2(\mathbb{Z})$
をとる。この時、$\gamma \cdot z = z +1, cz+d =1$となるため、
unrestricted modular form$f$に対し、$\gamma \cdot f(z)= f(z +1)=f(z)$となる。
よって、実軸に沿って周期関数となることがわかる。
\end{proof}

\begin{proof}{\bf{フーリエ係数が$y$方向に不変なこと}}


実軸に沿って周期的であり、上の関数は正則であるため、実軸にそってフーリエ展開したものが自身と一致する
つまり、\ref{thm:fourier}を使うことで、$f(x + iy)= \displaystyle  \sum_{n= - \infty }^{\infty}a_n(y)e^{2 \pi i x n}$となる。\\
ただし、$a_n(y) e^{2 \pi i x n}= \int_{-\frac{1}{2}}^{\frac{1}{2}}f(x+iy)e^{-2 \pi i (x+iy) n}dx$となる。
式をいくつか変形することで、
\begin{equation*}
 f(x+iy)= \sum_{n= - \infty }^{\infty}a_n(y)e^{2 \pi n y} e^{2 \pi i (x+iy) n}
\end{equation*}
とかける。$a_n(y)e^{2 \pi n y}= \int_{-\frac{1}{2}}^{\frac{1}{2}}f(x+iy)e^{-2 \pi in(x+iy)}dx$が$y$に依存しないことを示す。

$y$方向に$\alpha$方向ずらした長方形に対し、
$f(z)e^{2 \pi in z}$は正則になるので
$\oint f(x+iy)e^{-2 \pi in(x+iy)}dz = 0$となる。(画力でなく、図形がなく、申し訳ない。)

$f(x + iy ) = f( x + 1 + iy)$となるので、
$\int_{y}^{y+ \alpha}f(-1/2 + iy)dy =\int_{y}^{y+ \alpha}f(1/2 + iy)dy $となる。
よって、
$\int_{-\frac{1}{2}}^{\frac{1}{2}}f(x+iy)e^{-2 \pi in(x+iy)}dx =
\int_{-\frac{1}{2}}^{\frac{1}{2}}f(x+i(y+\alpha))e^{-2 \pi in(x+i(y+\alpha))}dx$
となる。任意の$\alpha$に対して、成り立つので、題意は示された。
\end{proof}


\subsubsection{Definition of Modular form}
\label{subs:Definition of Modular form}


$\mathcal{H} \to \mathbb{C}, z \mapsto  e^{2 \pi i z}$を考えると、$z \to \infty$の時、$e^{2 \pi i z} \to 0$となる。
これから、上のフーリエ展開は、$z = \infty$でのローラン展開と考えられる。
すると、自然に興味が出てくるのが、$z = \infty$が除去可能特異点かどうかである。
除去可能特異点である時、つまり、$z = \infty$で微分可能であるときだけをmodular formとよび、調べる。
正確にmodular formを定義する。
\begin{dfn}
  正則関数$f:\mathcal{H} \to \mathbb{C}$は以下を満たすとき、weight $k$のmodular formという。
  \begin{itemize}
    \item   $\forall  \gamma=
      \begin{pmatrix}
        a & b \\
        c & d \\
      \end{pmatrix}
      \in \Gamma(N),  \gamma \cdot f(z) = (cz + d)^{k}f(z)$
    \item $f(z) = \displaystyle_{n= 0}^{\infty}c_nq^n$ ただし、$q = e^{2 \pi i z}$
  \end{itemize}
  また、$c_0 = 0$の時、$f$はweight $k$ のcusp formという。
\end{dfn}

modular formが定義できたので、実際に何か例を考えよう。

\begin{ex}[Eisenstein series]
 \begin{equation*}
   f(z)=\sum_{n,m=1}^{ \infty}\frac{1}{(n+mz)}^{2k}
 \end{equation*}
これはweight $2k$の保型形式になっている。($k \ge 2$)
\end{ex}

保系形式になることを以下の手順で確認する
\begin{enumerate}
  \setlength{\parskip}{0cm} % 段落間
  \setlength{\itemsep}{0cm} % 項目間
  \item 任意の$\gamma$について$\gamma \dot f(z) = (cz+d)^{2k}f(z)$となること
  \item $q$展開の表示
\end{enumerate}
証明は略。

% ==================================
% 保系形式の性質
% ==================================
\subsubsection{L 関数}
\label{sub:L 関数}
保系形式のの$L$関数を以下で定義する。
$f= \sum_{n= -\infty}^{\infty}c_n q^n$となる時、
$L(s,f)= \sum_{n=1}^{ \infty } \frac{c_n}{n^s}$

\begin{rem}
 この保型形式は関数等式を満たすため、解析接続できる。
\end{rem}

\section{algebra of modular form}
\label{sec:algebra of modular form}

% ==========================
% ベクトル空間となること
% ==========================


% ==========================
% ベクトル空間としての決定
% ==========================

\section{Peterson innner prodcut and Hecke operator}
\label{sec:Peterson innner prodcut and Hecke operator}


% ==========================
% Peterson内積
% ==========================



% ==========================
% Hecke作用素
% ==========================

% ==========================
% 同時固有関数
% ==========================

\section{old forrfm/new form}
\label{sec:old forrfm/new form}



\section{Definition of Elliptic Curve}
\label{sec:Definition of Elliptic Curve}
楕円曲線を定義し、性質について述べる。
証明は基本的に記述しない。

% =========================
% 代数多様体の定義
% =========================
\subsection{Algebraci Variety}
\label{sub:Algebraci Variety}
楕円曲線は代数多様体の一種である。そのため、楕円曲線の定義に触れる前に最低限の代数多様体の定義について述べる。
$K$を体とし、$\bar{K}$を分離閉包とする。
$\mathbb{A}^n:=\bar{K}^n$

$I$を$\bar{K}[X_1,\dots,X_n]$のイデアルとする。
多項式環はネーターなので、$I$は有限生成となる。
つまり$I$は$f_1,\dots,f_n$を用いて、$I = \{ \sum g_if_i | g_i \in \bar{K}[X_1,\dots,X_n] \}$とかける。
このとき、$V(I) \subset \bar{K}$を以下で定義する。
\begin{equation*}
  V(I) = \{ (x_1 , \dots , x_n) \in \bar{K}^n | f_i(x_1,\dots,x_n)=0 \}
\end{equation*}

% =========================
% ワイエルシュトラス多項式
% =========================
\subsection{Weirstrass Polynomial}
\label{sub:Weirstrass Polynomial}

Projevetive Varietyとして楕円曲線を定義する。
まず、その定義方程式になる、ワイエルシュトラス多項式について説明する。
F(X,Y,Z)=
f(x,y)=

% =========================
% 楕円曲線の定義
% =========================
\begin{dfn}
 $E/K$が$K$上の楕円曲線であるとは、$K$係数ワイエルシュトラス多項式の解全体のなす、一次元非特異代数曲線であること。
 すなわち、ある$\Delta \neq 0$となる、ワイエルシュトラス多項式$F$に対し、
 $E = \{(x,y,z) \in P^3(K) |F(x,y,z)=0 \} $となること。
\end{dfn}

% =========================
% 楕円曲線の性質
% =========================
isogenyを定義する。

% =========================
% C上の楕円曲線
% =========================

保型形式をとれば、楕円曲線が定義できる。



\section{Eichler-Shimura theory}
\label{sec:Eichler-Shimura theory}
Eichler-Shimura theoryについて解説する。

\subsubsection{Abelian Variety and Jacobian Variety}
\label{subs:Abelian Variety and Jacobian Variety}
Aがabelian varietyとは、nonsigular projective variety/$\mathbb{C}$ with a distinguished point $O$ and with an abelian group $C$ such that for any hom $F'':A \to A''$, ker$F'' \supset C$


\begin{thm}{[k],[11.74 Eichler-Shimura]} \\
  $f(\tau)=\sum_{n=1}^{\infty}c_ne^{2 \pi i n \tau}$を$S_2(\Gamma_0(N))$のnew formとする。
  $c_1=1$。、$c_n \in \mathbb{Z}$となるように正規化したものとする。
  この時、以下を満たすpair $(E, \nu)$が存在する
  \begin{enumerate}
    \item $E$ は $\mathbb{Q}$上の楕円曲線であり、$(E, \nu)$は$\mathbb{Q}$上のアーベル多様体$J$を部分多様体$A$で割ったものとなる。
    \item $t(n) \in  \mathrm{End}(J)$の作用は$A$に対して、stableであるため、$E$への作用が定義され、それは$c_n$倍写像に一致する。
    \item $\mu(f)$は$\nu^*( \omega)$の指数。ただし、$ \omega$は$E$の不変微分
    \item $\Lambda_f  = \{ \Phi_f(r) = \int_{\tau_0}^{\gamma(\tau_0)}$が成り立つならば、$\Lambda_f$は$\mathbb{C}$のlatticeであり、$E$は$\mathbb{C}$上$\mathbb{C}/\Lambda_f$と同型。
    \item $E$の$L$関数は有限個の素点を除いて、オイラー積が$f$と一致する。
  \end{enumerate}

\begin{proof}
\begin{lem}
$t(n) \in \mathrm{End}(J)$は$\mathbb{Q}$上定義されている。
\end{lem}
上記を一度認めて、議論する。
$T$を.$\mathrm{End}_{\mathbb{Q}}(J)$のcommutative $\mathbb{Q}$ subalgebraで$t(n)$によって生成されたとする。
$t(n)$を $g \times g$ matrix pf subalgebra
と同一視できできるので、$T$は $M(g,\mathbb{Q})$の subalgebraとなり、有限次元であることがわかる。
Wedburmの構造定理より、
of $M(g,\mathbb{Q})$なので、
\end{proof}










\end{thm}


\begin{thebibliography}{数字}
  \bibitem{K} Elliptic Curves.・Anthony W. Knapp
  \bibitem{M} Modular Functions and Modular Forms・J. S. Milne
  \bibitem{S1} Arithmetic of Elliptic Curvers
  \bibitem{S2}
  \bibitem{??} http://math.umn.edu/~brubaker/docs/mit/785notes.pdf
\end{thebibliography}

\end{document}
