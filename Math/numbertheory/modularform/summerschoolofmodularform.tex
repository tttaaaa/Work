%===============
%一行目に必ず必要
%文章の形式を定義
%===============
\documentclass{ujarticle}
%===============
%パッケージの定義、必要か不明
%===============
%この下4つを加えることで、mathbbが機能した
\usepackage{amsthm}
\usepackage{amsmath}
\usepackage{amssymb}
\usepackage{amsfonts}
%可換図式用パッケージ
\usepackage{amscd}
\usepackage[all]{xy}
\usepackage{tikz-cd}
%リンク用パッケージ
\usepackage[dvipdfmx]{hyperref}
%複数行コメント
%\usepackage{comment}

%タイトルデータ
\title{保型形式夏の学校}
\author{take}
\date{2016/August}


%===============
%定理環境の設定
%セクション毎
%===============
\newtheorem{thm}{Theorem}[section]
\newtheorem{dfn}[thm]{Definition}
\newtheorem{prop}[thm]{Propostion}
\newtheorem{lem}[thm]{Lemma}
\newtheorem{ex}[thm]{Example}
\newtheorem*{prob}{Problem}
\newtheorem*{rem}{Remark}
\newtheorem{prf}{Proof}

\begin{document}
\section{introduction}
\label{sec:introduction}
数学好きだが、大学の数学はあまり知らない人を想定して、保型形式と楕円曲線の初歩について解説する。
今回の目的は以下2つである。
\begin{enumerate}
  \item 複素関数として定義できる保型形式が数論とつながりがあること
  \item 連綿と連なる理論を概観し、数論のすごさを1\%でも理解すること
\end{enumerate}
ここで簡単に解説する。
保型形式は、$SL(2,\mathbb{Z})$が特殊な形で作用する上半平面$\mathcal{H}$上の正則関数として定義される。ただの複素関数では数論とつながりを見えづらいが、保型形式、特にカスプ形式は$L$関数が定義され、Euler積を持つ。$\zeta$関数を用いて、素数定理が示されて以降、様々な$\zeta$,$L$関数が作られ、それらを用いて数論的な性質が調べられないか考えられきた。例えばガロア群の指標から作るArtin $L$関数、ディリクレ$L$関数の$p$進類似である、p進$L$関数などである。
また、保型形式は数論的に興味深い楕円曲線とも深くつながりがある。
例えば、$\mathbb{C}$上の楕円曲線からは楕円関数が自然に定義され、それらは保型形式になっている。


楕円曲線と保型形式の関係を見るにも、複素解析の正則性や留数などから出てくる積分の計算、また、積分や関数の掛け算を線形代数的に捉えることにより、保型形式全体のなす空間を代数的に捉えることができるようになる。
その一方で、楕円曲線を定義するには、可換環、(射影)代数多様体の理論を使い、これらを結びつける楕円関数はあるコンパクトリーマン面の正則関数となっている。こうした様々な道具を多彩に使い、一つの問題に使われていることに何らかのすごさを感じてくれると幸いである。

さらに深い結果として、保型形式の特殊なクラスから、$\mathbb{Q}$上の楕円曲線が作られることがわかっている(Eichler-Shimura theory)。逆に、楕円曲線から保型形式が作られることもわかっている(Taniyama-Shimura Conjecture)。これらを用いてかの有名なFeramat's Last Theoremが証明される。
本記事では、Eichler-ShimuraもTaniyama-Shimuraも立ち入らないが、楕円曲線を通じて、数論の深遠さを少しでも感じてもらいたい。

具体的な内容としては、楕円曲線と保型形式を定義し、楕円関数が保型形式になっていること、また、保型形式全体がベクトル空間になっていることをみる。
\section{射影代数多様体}
\label{sec:射影代数多様体}

初学者向けなので、体とは$\mathbb{C}$のこととする。
まず、射影空間を定義する。
\begin{dfn}
  $(x_0,\dots,x_n) \in \mathbb{C}^n\\{(0,\dots,0)} $を以下の同値関係で割ったものを射影空間$\mathbb{P}^n(\mathbb{C})$という。
  $(x_0,\dots ,x_n) \sim (y_0,\dots,y_n)$とは、とは、ある$\lambda \in \mathbb{C}^{\times}$が存在し、$(x_0,\dots ,x_n)= (\lambda y_0,\dots, \lambda y_n)$とかけること。
\end{dfn}
これが同値関係になっていること、および、同値関係を定めれば商集合が作れることは既知とする。



\end{document}
