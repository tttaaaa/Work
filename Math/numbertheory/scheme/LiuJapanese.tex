%===============
%一行目に必ず必要
%文章の形式を定義
%===============
\documentclass{ujarticle}
%===============
%パッケージの定義、必要か不明
%===============
%この下4つを加えることで、mathbbが機能した
\usepackage{amsthm}
\usepackage{amsmath}
\usepackage{amssymb}
\usepackage{amsfonts}
%可換図式用パッケージ
\usepackage{amscd}
\usepackage[all]{xy}
\usepackage{tikz-cd}
%リンク用パッケージ
\usepackage[dvipdfmx]{hyperref}
%複数行コメント
%\usepackage{comment}

%タイトルデータ
\title{p進解析入門}
\author{ari}
\date{2016/9/22}


%===============
%定理環境の設定
%セクション毎
%===============
\newtheorem{thm}{Theorem}[section]
\newtheorem{dfn}[thm]{Definition}
\newtheorem{prop}[thm]{Propostion}
\newtheorem{lem}[thm]{Lemma}
\newtheorem{cor}[thm]{Corllary}
\newtheorem{epl}[thm]{Example}
\newtheorem*{prob}{Problem}
\newtheorem*{rem}{Remark}
\newtheorem{prf}{Proof}

\begin{document}

% タイトルを出力
\maketitle
% 目次の表示
\tableofcontents


スキームの勉強として,Liuを日本語訳しながら,証明を書く.

\section{Some topics in commutative algebra}
\label{sec:Some topics in commutative algebra}

\section{General Proeperties of Schemes}
\label{sec:General Proeperties of Schemes}
この章ではスキームの基礎理論を紹介する.最初の3節を用い,例を交えながら環と環の射を定義する.
その後,2.4と2.5ではスキームの位相的な性質について議論する.

\subsection{Spectram of a ring}
\label{sub:Spectram of a ring}

位相多様体や微分多様体は$\mathbb{R}^n$の開集合によるlocalなchartを貼り合わせることで,構成されている.
スキームも同様で,スキームの場合は貼り合わせるものはアフィンスキームと言われる.この章ではアフィンスキームの下部構造
である位相構造について定義する.また代数幾何の直感を養うために,algebraic setについても調べる.

\subsubsection{Zariski topology}
\label{subs:Zariski topology}

********************
** subsection全体を飛ばす.
********************

\subsection{Ringed topological spaces}
\label{sub:Ringed topological spaces}
スキームを定義する一つの方法として,局所的にアフィンスキームと同型な環付き空間として定義する方法がある.
環付き空間の議論の前に層の理論を軽く解説する.層は代数幾何で絶対必要な道具であり,
局所的なデータを集めて,大域的なデータを作ることができる.

\subsubsection{Sheaves}
\label{subs:Sheaves}
位相空間上の層について定義と局所的な性質について記載する.
層の理論の詳細を知りたい人は[40]などを参照せよ.

\begin{dfn}
 $X$を位相空間とする.abelian groupのpresheaf $\mathcal{F}$を以下のデータで構成される.
 \begin{enumerate}
   \item $X$の任意の開集合$U$に対し,アーベル群$\mathcal{F}(U)$
   \item $V \subset U$に対し,$\rho_{UV}:F(U) \to F(V)$を定める.
 \end{enumerate}
 さらに,以下の3条件を満たす.
 \begin{enumerate}
   \item $\mathcal{F}(\emptyset)=0$
   \item $\rho_{UU}=id$
   \item $W \subset V \subset U$に対し,$\rho_{UW}=\rho_{VW} \circ \rho_{UV}$となる.
 \end{enumerate}
 $s \in \mathcal{F}(U)$を$U$のセクションという.$s|_{V}$で$\rho_{UV}(s)$を表す.
\end{dfn}
\begin{dfn}
 $\mathcal{F}$が層であるとは,上記に加え,以下を満たすものである.
 \begin{enumerate}
   \item $U$は$X$の開集合で,$s \in \mathcal{F}(U),\{U_i\}$を$U$の被覆とする.
   任意の$i$に対し,$s|_{U_i}=0$となる場合,$s=0$
   \item 上と表記は同じで,$s_i \in \mathcal{F}(U_i)$をopen coveringとする.
   $s_i|_{U_i \cap U_j} = s_j|_{U_i \cap U_j}$が成り立っていたとすると,
   $s \in \mathcal{F}(U)$の元で$s|_{U_i}=s_i$となるものが取れる.
 \end{enumerate}
\end{dfn}
環の層や代数の層,また,部分層の概念も同様に定義できる.

\begin{epl}
 $X$を位相空間とする.任意の$X$の開部分集合$U$に対し,$\mathcal{C}(U)=C^0(U,\mathbb{R})$とする.
 $\rho_{UV}$として実際の関数の制限を取る.この時$\mathcal{C}$が層となる.
\end{epl}

\begin{epl}
 $A$をnon-trivialなアーベル群とする.$X$を位相空間とし,$U$を開集合とする.
 $\mathcal{A}_{X}(U)=A$,$U,V$が空集合でなければ,$\rho_{UV}=id$とする.
 $\mathcal{A}_X$はpresheafであるが,一般にはsheafではない.
 例えばdisjointは開集合$SU,V$が取れたとすると$U \cap V =\emptyset$となる.
 異なる$s_1,s_2 \in A$をとると,$s_1|_{U\cap V}=s_2|_{U \cap V}=0$となるので,矛盾する.
 層にするには連結成分毎に$A$で連結成分を複数個の場合はその直和になる必要がある.*訳者補足.
\end{epl}

\begin{rem}
 **後で
\end{rem}

\section{Morphisms and base change}
\label{sec:Morphisms and base change}
  ================TBD=====================

\section{Some local properties}
\label{sec:Some local properties}

\subsection{Normal schemes}
\label{sub:Normal schemes}

  ================TBD=====================

\subsection{Regulara schemers}
\label{sub:Regulara schemers}

代数幾何において,局所的な構造だけを考えると,最も単純なものがregular scheme $X$である.
regular schmerはある意味において,affineなものと近い.それは,局所環$O_{X,x}$のformal completin
の構造(Proposition 2.27)や$O_{X,x}$の代数的な構造(Theorem 2.16)を通して,理解される.
最初のsubsectionでは,スキームの接空間を定義し,そこから,regularを定義し,Jacobian critetrion
(Theorem 2.19)を示す.

\subsubsection{Tanget space to a scheme}
\label{subs:Tanget space to a scheme}
\begin{dfn}
$X$をスキームとし,$x \in X$とする.$\mathfrak{m}_x$を$O_{X,x}$の極大イデアルとし$k(x)$
をその剰余体とする.この時,$\mathfrak{m}_x/{\mathfrak{m}_x}^2=\mathfrak{m}_x$は自然に
$k(x)$ベクトル空間となる.その双対${\mathfrak{m}_x/{\mathfrak{m}_x}^2}^{\vee}=
\mathrm{Hom}(\mathfrak{m}_x/{\mathfrak{m}_x}^2,k(x))$を\bf{Zariski tangent space}という
\end{dfn}
  ================TBD=====================

\subsection{Flat morphisms and smoot morphism}
\label{sub:Flat morphisms and smoot morphism}

\subsubsection{Flat morphism}
\label{subs:Flat morphism}

\subsubsection{Etale morphisms}
\label{subs:Etale morphisms}

\subsubsection{Smoot morphism}
\label{subs:Smoot morphism}
\begin{dfn}
 $X$を体$k$上の代数多様体とし,$\bar{k} $を $k$の代数閉体とする.
 $X$が$x \in X$で\bf{smooth}とは, $x$の上にある $X_{\bar{k}}$の点が正則なら・・・・????
\end{dfn}

\end{document}
