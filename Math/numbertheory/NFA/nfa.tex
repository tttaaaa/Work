%===============
%一行目に必ず必要
%文章の形式を定義
%===============
\documentclass{ujarticle}
%===============
%パッケージの定義、必要か不明
%===============
%この下4つを加えることで、mathbbが機能した
\usepackage{amsthm}
\usepackage{amsmath}
\usepackage{amssymb}
\usepackage{amsfonts}
%可換図式用パッケージ
\usepackage{amscd}
\usepackage[all]{xy}
\usepackage{tikz-cd}
\usepackage{my-default}
%\usepackage{kmathmacro}
%リンク用パッケージ
%\usepackage[dvipdfmx]{hyperref}
%複数行コメント
\begin{document}
このPDFではSCHNEIDER先生のNFAの訳を適宜つけていく.
目標はp進関数解析を理解すること.特に関数空間の位相になれ,通常の関数解析をしている理論に対し,
p進関数解析で改めて作り直すことを目指す.
\begin{rem}
 記号のテスト.
 $\tl{B}$
 $\im$ ABC
\end{rem}

\part{Foundations}
\label{chap:Foundations}

\section{Non Archmedian Fields}
\label{sec:Non Archmedian Fields}
$K$を体とする.$K$上の絶対値とは,関数




\end{document}
