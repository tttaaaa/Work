%===============
%一行目に必ず必要
%文章の形式を定義
%===============
\documentclass{ujarticle}
%===============
%パッケージの定義、必要か不明
%===============
%この下4つを加えることで、mathbbが機能した
\usepackage{amsthm}
\usepackage{amsmath}
\usepackage{amssymb}
\usepackage{amsfonts}
%可換図式用パッケージ
\usepackage{amscd}
\usepackage[all]{xy}
\usepackage{tikz-cd}
%リンク用パッケージ
\usepackage[dvipdfmx]{hyperref}
%複数行コメント
%\usepackage{comment}


%===============
%定理環境の設定
%セクション毎
%===============
\newtheorem{thm}{Theorem}[section]
\newtheorem{dfn}[thm]{Definition}
\newtheorem{prop}[thm]{Propostion}
\newtheorem{lem}[thm]{Lemma}
\newtheorem{ex}[thm]{Example}
\newtheorem*{prob}{Problem}
\newtheorem*{rem}{Remark}
\newtheorem{prf}{Proof}

\begin{document}
情報幾何の理論理解を目標に多様体、バンドル、接続など、微分幾何の基礎知識を学習する。

\section{多様体の基礎概念}
\label{sec:多様体の基礎概念}

\section{ベクトルバンドル}
\label{sec:ベクトルバンドル}

\section{1の分割}
\label{sec:1の分割}

\section{埋め込みとはめ込み}
\label{sec:埋め込みとはめ込み}


\section{接続とリーマン計量}
\label{sec:接続とリーマン計量}

測地線

\section{双対アフィン幾何学}
\label{sec:双対アフィン幾何学}

\begin{thm}
 双対平坦多様体$(M,g,\nabla,\nabla^{*} )$
\end{thm}


\end{document}
