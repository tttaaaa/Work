\documentclass[utf8]{beamer}
\usetheme{CambridgeUS}

%===============
%この下4つを加えることで、mathbbが機能した
\usepackage{amsthm}
\usepackage{amsmath}
\usepackage{amssymb}
\usepackage{amsfonts}
%可換図式用パッケージ
\usepackage{amscd}
\usepackage[all]{xy}
\usepackage{tikz-cd}
%リンク用パッケージ
%\usepackage[dvipdfmx]{hyperref}
%複数行コメント
%\usepackage{comment}


%===============
%定理環境の設定
%セクション毎
%===============
\newtheorem{thm}{Theorem}
\newtheorem{dfn}[thm]{Definition}
\newtheorem{prop}[thm]{Propostion}
\newtheorem{lem}[thm]{Lemma}
\newtheorem{ex}[thm]{Example}
\newtheorem*{prob}{Problem}
\newtheorem*{rem}{Remark}
\newtheorem{prf}{Proof}

\title{はじめてのBeamer}
\author{高木貞治}
\date{\today}


\begin{document}
\maketitle
\begin{frame}{スライド}
 スライドができたよ!やったね!
\begin{thm}
 (可換図式や代数、位相に慣れている人向けの)別証明
\end{thm}
\end{frame}


\section{はじめに}
\begin{frame}{何を問題としているか}
\begin{itemize}
\item こんなこと
\item あんなこと
\item しかも\alert{そんなことまで}
\end{itemize}
\end{frame}


\begin{frame}{何を問題としているか}
\xymatrix{
\mathbb{C}  \ar[r]_{f_{\alpha}} \ar[d]^{p_{\Gamma}} & \mathbb{C} \ar[d]_{p_{\alpha \Gamma}} \\
\mathbb{C}/\Gamma \ar[r]^{\tilde{f_{\alpha}}} & \mathbb{C}/{\alpha \Gamma}  \ar[d]_{p_{ \Gamma^{'}}} \\
 & \mathbb{C}/\Gamma^{'} }
\end{frame}

\begin{frame}{スライド}
\begin{enumerate}
\item 箇条書きと一緒に使うと
\item 効果的なプレゼンが
\item 可能となる\structure{ブロックの領域}が装飾される
\end{enumerate}
\end{frame}

\begin{frame}{ブロック環境を意味のまとまりとして使う}
  \begin{block}{概略}
    \begin{itemize}
      \item usethemeでテーマを指定すると\structure{ブロックの領域}が装飾される
      \item block, alertblock, exampleblock環境がある
    \end{itemize}
  \end{block}
  \begin{alertblock}{注意}
    \begin{itemize}
      \item blockタイトルないと\alert{コンパイルエラー}になる
    \end{itemize}
  \end{alertblock}
\end{frame}





\end{document}