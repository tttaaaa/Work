%===============
%一行目に必ず必要
%文章の形式を定義
%===============
\documentclass{ujarticle}
%===============
%パッケージの定義、必要か不明
%===============
%この下4つを加えることで、mathbbが機能した
\usepackage{amsthm}
\usepackage{amsmath}
\usepackage{amssymb}
\usepackage{amsfonts}
%可換図式用パッケージ
\usepackage{amscd}
\usepackage[all]{xy}
\usepackage{tikz-cd}
%リンク用パッケージ
\usepackage[dvipdfmx]{hyperref}
%複数行コメント
%\usepackage{comment}

%タイトルデータ
\author{ari}
\title{Projectvie variety}
\date{2016/12/6}


%===============
%定理環境の設定
%セクション毎
%===============
\newtheorem{thm}{Theorem}[section]
\newtheorem{dfn}[thm]{Definition}
\newtheorem{prop}[thm]{Propostion}
\newtheorem{lem}[thm]{Lemma}
\newtheorem{cor}[thm]{Corllary}
\newtheorem{epl}[thm]{Example}
\newtheorem*{prob}{Problem}
\newtheorem*{rem}{Remark}
\newtheorem*{yodan}{余談,疑問}
\newtheorem{prf}{Proof}

\begin{document}

\section{Introduction}
\label{sec:Introduction}
TeXは現在の数学界隈で非常に重要である.それはTeX以外に
数学でよく使われるギリシャ文字,特殊記号を出せるものがないからであろう.
一方で使いこなすにはなかなか癖の多いものである.そのため,自分の知識をまとめ,理解度を増やす.
特に自分がよく忘れてるところ,忘れるところと,同時にTeXで高速に作業を作るための方法を模索する.
それはプログラミング力,及び管理力を増やす.


\section{図全般}
\label{sec:図全般}
図の作成方法をまとめる.
以下を対象にする
\begin{description}
  \item[箇条書き]
  \begin{itemize}
    \item 箇条書きの種類
    \item 行間のサイズ調整
    \item 段落記号の変更
  \end{itemize}
  \item[表の調整]
  \item[グラフの描画]
  \item[可換図式の描画]
\end{description}

\subsection{箇条書き}
\label{sub:箇条書き}
TeXの段落には以下の3種類がある.
\begin{description}
  \item[箇条書き] itemize
  \item[数字つき] enumerate
  \item[記号付き箇条書き] description
\end{description}

表示が少し変わるという以外本質的に同じなので,共通の変更についてまず記載する.
個別に変わるところについて後で記載する.
これらはネストすることも可能である.



\section{キーワード}
\label{sec:キーワード}

\subsection{関数の作成}
\label{sub:関数の作成}

\subsection{Snippet}
\label{sub:Snippet}

\subsection{記号}
\label{sub:記号}



\end{document}
