%===============
%一行目に必ず必要
%文章の形式を定義
%===============
\documentclass{ujarticle}
%===============
%パッケージの定義,必要か不明
%===============
%この下4つを加えることで,mathbbが機能した
\usepackage{amsthm}
\usepackage{amsmath}
\usepackage{amssymb}
\usepackage{amsfonts}
%可換図式用パッケージ
%\usepackage{tikz}
\usepackage{amscd}
%\usepackage[all]{xy}
%\usepackage{tikz-cd}
%リンク用パッケージ
\usepackage[dvipdfmx]{hyperref}
%複数行コメント
%\usepackage{comment}
\usepackage[dvipdfmx]{graphicx}
% \def\pgfsysdriver{pgfsys-dvipdfmx.def}%(graphicxパッケージを使用しない場合)
\usepackage[dvipdfmx,svgnames]{xcolor}%tikzパッケージよりも前に読み込みます。
\usepackage{tikz}

%タイトルデータ
\title{Math-iine learning}
\author{ari}
\date{2017/04/04-}
%===============
%定理環境の設定
%セクション毎
%===============
\newtheorem{thm}{Theorem}[section]
\newtheorem{dfn}[thm]{Definition}
\newtheorem{prop}[thm]{Propostion}
\newtheorem{lem}[thm]{Lemma}
\newtheorem{cor}[thm]{Corllary}
\newtheorem{epl}[thm]{Example}
\newtheorem*{prob}{Problem}
\newtheorem*{rem}{Remark}
\newtheorem{prf}{Proof}
\newtheorem{clm}{Claim}
\newtheorem{hyp}{HYPOTHESIS}
\newtheorem{mtm}{Main Thoerem}




\begin{document}

\part{記号の準備}
\label{part:記号の準備}
ニューラルネットワークの例
\def\layersep{2.5cm}

\begin{tikzpicture}[shorten >=1pt,->,draw=black!50, node distance=\layersep]
    \tikzstyle{every pin edge}=[<-,shorten <=1pt]
    \tikzstyle{neuron}=[circle,fill=black!25,minimum size=17pt,inner sep=0pt]
    \tikzstyle{input neuron}=[neuron, fill=green!50];
    \tikzstyle{output neuron}=[neuron, fill=red!50];
    \tikzstyle{hidden neuron}=[neuron, fill=blue!50];
    \tikzstyle{annot} = [text width=4em, text centered]

    % Draw the input layer nodes
    \foreach \name / \y in {1,...,4}
    % This is the same as writing \foreach \name / \y in {1/1,2/2,3/3,4/4}
        \node[input neuron, pin=left:Input \#\y] (I-\name) at (0,-\y) {};

    % Draw the hidden layer nodes
    \foreach \name / \y in {1,...,5}
        \path[yshift=0.5cm]
            node[hidden neuron] (H-\name) at (\layersep,-\y cm) {};

    % Draw the output layer node
    \node[output neuron,pin={[pin edge={->}]right:Output}, right of=H-3] (O) {};

    % Connect every node in the input layer with every node in the
    % hidden layer.
    \foreach \source in {1,...,4}
        \foreach \dest in {1,...,5}
            \path (I-\source) edge (H-\dest);

    % Connect every node in the hidden layer with the output layer
    \foreach \source in {1,...,5}
        \path (H-\source) edge (O);

    % Annotate the layers
    \node[annot,above of=H-1, node distance=1cm] (hl) {Hidden layer};
    \node[annot,left of=hl] {Input layer};
    \node[annot,right of=hl] {Output layer};
\end{tikzpicture}
% End of code

\part{第1回-2016/}

\part{第2回-2016}
ランダムフーリエ?

\part{第3回-}
Topological Data Analysisの基本について解説する

\part{第4回-}
Discrete Morse Thoeryについて解説する.

\part{第5回-}
論文を紹介する.

\part{第6回-}
VAEに向け、NNの基本と変分法を解説する.

\section{今回話すこと}
\label{sec:今回話すこと}
VAEの理解を向けNNの基本的な手法を理解する.
\begin{itemize}
  \item FrequentとBayesian
  \item NNの知識
  \item 変分推論
  \item 生成モデルと識別モデル
\end{itemize}
\section{FrequentとBayesian}
\label{sec:FrequentとBayesian}
FrequentとBayesianの違いは不確かさの根源がどこにあるのかで区別できる.
\begin{itemize}
  \item[\textbf{Frequent}] データの方に不確かさがある.
この考えでは,ある正しいとされる仮説に対し,このデータは○○%の確率で得られると考える.
データは確率変数$X$で与えられ,$X=N(μ,σ^2)$と仮定し,$n$個サンプリングしたもの.
  \item[\textbf{Bayesian}] 不確かさは情報の不足に由来する.
この考えでは,この仮説が正しい確率は○○\%であると考える.
\end{itemize}

\begin{table}[htb]
  \centering
  \begin{tabular}{l|cc}
     & パラメータ & データ   \\ \hline
     F & 定数 & 確率変数 \\
    B& 確率変数 & 定数 \\
  \end{tabular}
\end{table}

\begin{epl}[コイントス]
コイントスを100回試行し、$H$が40回,$T$が60回出たとする.
\subsection{Frequentist}
\label{sub:Frequentist}
Frequencyの手法では,データは確率分布ではなく,有限集合と考える.
$D=\{H\} \times \{T\},…, \times \{T\}\ \subset {T,H}^{100} $とする.
\textbf{表が出る確率を$\theta$(で互いに独立)とする.}
(これは現実の現象をモデル化するための仮定であり,$\theta$はまだわからないので,未定)
$D$となる確率は
\begin{align*}
P(D|\theta)=\theta^40(1-\theta)^60 \\
\mathrm{log}p(D|\theta)=40\mathrm{log}\theta +60\mathrm{log}(1-\theta)
\end{align*}
となる.
100回試行して,上の現象が起きたので,この現象が起きる確率が最も高い$\theta$が
尤もらしい.
上の確率が一番高くなる$\theta$,
つまり,$p(D|\theta)$をmaximiseする$\theta$は
微分して計算すると$\theta=4/10$となることがわかる.

Bayesian $\theta$の上の確率分布として,どの$\theta$が確からしいか?
Dを観測する前 $p(\theta)=1(0 <=\theta<=1)$(事前分布)
これは,Beta(1,1)
Beta(1/2,1/2)等も関係ある?

$X_i$:i番目のコイントススの結果
 $p(\theta|x_1)$を計算する
 \begin{equation*}
 p(\theta|x_1)=p(\theta,x_1)/p(x_1)=p(x_1|\theta)p(\theta)/∫p(x1|\theta)p(\theta)d\theta
=
\end{equation*}
\end{epl} コイントス:表(Head)の確率


確率測度の全空間を取替ることが多いが,それを隠蔽している.


\part{第7回-2017/3/18}
\label{part:第7回-2017/3/18}

Variational Autoencoder\cite{Kingma Welling}を理解する.

\section{今回の話すこと}
\label{sec:今回の話すこと}
前回話した変分推論の考え方に基づき,$L(\theta,\phi)$を$\theta,\phi$に対して最大化する.
このために,以下について説明する
\begin{itemize}
  \item NNで確率を表現する
  \item Backporpagation
  \item VAEの学習
\end{itemize}
\begin{rem}
 VAEは画像生成の方法(モデル)の一つであり,他にはGAN等がある.
 画像生成には学習用データを増やしたいという意図があるが,これについては学習済みのものからデータを生成するため,
 本質的にデータが増えておらず,意味がないと考えている人も多い.
\end{rem}
\section{NNで確率を表現する}
\label{sec:NNで確率を表現する}



\begin{thebibliography}{9}
  \bibitem{Kingma Welling} Diederik P Kingma, Max Welling,
    ``Auto-Encoding Variational Bayes '' https://arxiv.org/abs/1312.6114 2013.
  \bibitem{susan} S. M. Smith and J. M. Brady,
    ``SUSAN|A new approach to low level image processing,'' Int. J. Comput.
    Vis., vol.23, no.1, pp.45-78, May 1997.
\end{thebibliography}

\part{第8回-2017/4/15}
VC dimensionなど

\part{第9回-2017/4/29}

\part{第10回-2017/6/3}
方針及びSanovの定理.
\end{document}
