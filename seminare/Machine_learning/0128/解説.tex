%===============
%一行目に必ず必要
%文章の形式を定義
%===============
\documentclass{ujarticle}
%===============
%パッケージの定義、必要か不明
%===============
%この下4つを加えることで、mathbbが機能した
\usepackage{amsthm}
\usepackage{amsmath}
\usepackage{amssymb}
\usepackage{amsfonts}
%可換図式用パッケージ
\usepackage{amscd}
\usepackage[all]{xy}
\usepackage{tikz-cd}
%リンク用パッケージ
\usepackage[dvipdfmx]{hyperref}
%複数行コメント
%\usepackage{comment}
%MathFOnt
\usepackage{mathrsfs}



%タイトルデータ
\author{ari}
\title{p-adic Weil予想入門}
\date{2017/1/29}


%===============
%定理環境の設定
%セクション毎
%===============
\newtheorem{thm}{Theorem}[section]
\newtheorem{dfn}[thm]{Definition}
\newtheorem{prop}[thm]{Propostion}
\newtheorem{lem}[thm]{Lemma}
\newtheorem{cor}[thm]{Corllary}
\newtheorem{epl}[thm]{Example}
\newtheorem*{prob}{Problem}
\newtheorem*{rem}{Remark}
\newtheorem*{yodan}{余談,疑問}
\newtheorem{prf}{Proof}

%==================
%botスタイル
%=================
\setlength{\topmargin}{0cm}
\setlength{\oddsidemargin}{0.3cm}
\setlength{\evensidemargin}{0.3cm}
\setlength{\textwidth}{14.9cm}
\setlength{\textheight}{22.0cm}
\setlength{\headheight}{0.0cm}

\usepackage{amscd}
\usepackage{amsfonts}
\usepackage{amsmath}
\usepackage{amssymb}
\usepackage{amsthm}
\usepackage{ascmac}
\usepackage[T1]{fontenc}
\usepackage{here}
\usepackage{mathrsfs}
%持っていないためかエラーになった.
%\usepackage{slashbox}
\usepackage{txfonts}
\usepackage[all]{xy}

\allowdisplaybreaks
\renewcommand{\thefootnote}{\fnsymbol{footnote}}
\renewcommand{\contentsname}{Contents}
\renewcommand{\refname}{References}
\renewcommand{\indexname}{Index}
\renewcommand{\figurename}{Figure}
\renewcommand{\tablename}{Table}

%定理環境
%\theoremstyle{plain}
%\newtheorem{thm}{Theorem}[section]
%\newtheorem{lmm}[thm]{Lemma}
%\newtheorem{prp}[thm]{Proposition}
%\newtheorem{crl}[thm]{Corollary}
%\theoremstyle{definition}
%\newtheorem{dfn}[thm]{Definition}
%\newtheorem{rmk}[thm]{Remark}
%\newtheorem{exm}[thm]{Example}
%\newtheorem{qst}[thm]{Question}

% \setcounter{section}{-1}
% \renewcommand{\thefootnote}{\fnsymbol{footnote}}

%意味がわかっていない.考える必要あり.
\def\ens#1{\mathchoice{\left\{ #1 \right\}}{\{ #1 \}}{\{ #1 \}}{\{ #1 \}}}
\def\set#1#2{\mathchoice{\left\{ #1 \middle| #2 \right\}}{\{ #1 \mid #2 \}}{\{ #1 \mid #2 \}}{\{ #1 \mid #2 \}}}
\def\r#1{\text{\rm #1}}
\def\t#1{\text{#1}}
\def\v#1{\mathchoice{\left| #1 \right|}{| #1 |}{| #1 |}{| #1 |}}
\def\n#1{\mathchoice{\left\| #1 \right\|}{\| #1 \|}{\| #1 \|}{\| #1 \|}}
\def\rt#1#2{\sqrt[#1]{\mathstrut #2} \ }
\def\ol#1{\overline{#1}{}}
\def\tl#1{\tilde{#1}{}}
\def\ul#1{\underline{#1}{}}
\def\wh#1{\widehat{#1}{}}
\def\wt#1{\widetilde{#1}{}}
\newcommand{\longhookrightarrow}{\lhook\joinrel\longrightarrow}
\newcommand{\longtwoheadrightarrow}{\relbar\joinrel\twoheadrightarrow}
\newcommand{\im}{\r{im}}
\newcommand{\coker}{\r{coker}}
\newcommand{\coim}{\r{coim}}
\newcommand{\reprod}{\mathrlap{coprod}\prod}

\makeatletter
\newcommand{\colim@}[2]{
  \vtop{\m@th\ialign{##\cr
    \hfil $#1\operator@font colim$ \hfil\cr
    \noalign{\nointerlineskip\kern1.5\ex@}#2\cr
    \noalign{\nointerlineskip\kern-\ex@}\cr}}
}

\newcommand{\colim}{
  \mathop{\mathpalette\colim@{\rightarrowfill@\textstyle}}\nmlimits@
}
\makeatother

\newcommand{\address}{
    \footnote{
      あどれす
    }
}

\newcommand{\info}[2]{
  \footnote[0]{
    $
    \begin{array}{l}
      \r{MSC2010: #1} \\
      \r{Key words: #2}
    \end{array}
    $
  }
}

\newcommand{\bA}{\mathbb{A}}
\newcommand{\bB}{\mathbb{B}}
\newcommand{\bC}{\mathbb{C}}
\newcommand{\bD}{\mathbb{D}}
\newcommand{\bE}{\mathbb{E}}
\newcommand{\bF}{\mathbb{F}}
\newcommand{\bG}{\mathbb{G}}
\newcommand{\bH}{\mathbb{H}}
\newcommand{\bI}{\mathbb{I}}
\newcommand{\bJ}{\mathbb{J}}
\newcommand{\bK}{\mathbb{K}}
\newcommand{\bL}{\mathbb{L}}
\newcommand{\bM}{\mathbb{M}}
\newcommand{\bN}{\mathbb{N}}
\newcommand{\bO}{\mathbb{O}}
\newcommand{\bP}{\mathbb{P}}
\newcommand{\bQ}{\mathbb{Q}}
\newcommand{\bR}{\mathbb{R}}
\newcommand{\bS}{\mathbb{S}}
\newcommand{\bT}{\mathbb{T}}
\newcommand{\bU}{\mathbb{U}}
\newcommand{\bV}{\mathbb{V}}
\newcommand{\bW}{\mathbb{W}}
\newcommand{\bX}{\mathbb{X}}
\newcommand{\bY}{\mathbb{Y}}
\newcommand{\bZ}{\mathbb{Z}}
\newcommand{\cA}{\mathscr{A}}
\newcommand{\cB}{\mathscr{B}}
\newcommand{\cC}{\mathscr{C}}
\newcommand{\cD}{\mathscr{D}}
\newcommand{\cE}{\mathscr{E}}
\newcommand{\cF}{\mathscr{F}}
\newcommand{\cG}{\mathscr{G}}
\newcommand{\cH}{\mathscr{H}}
\newcommand{\cI}{\mathscr{I}}
\newcommand{\cJ}{\mathscr{J}}
\newcommand{\cK}{\mathscr{K}}
\newcommand{\cL}{\mathscr{L}}
\newcommand{\cM}{\mathscr{M}}
\newcommand{\cN}{\mathscr{N}}
\newcommand{\cO}{\mathscr{O}}
\newcommand{\cP}{\mathscr{P}}
\newcommand{\cQ}{\mathscr{Q}}
\newcommand{\cR}{\mathscr{R}}
\newcommand{\cS}{\mathscr{S}}
\newcommand{\cT}{\mathscr{T}}
\newcommand{\cU}{\mathscr{U}}
\newcommand{\cV}{\mathscr{V}}
\newcommand{\cW}{\mathscr{W}}
\newcommand{\cX}{\mathscr{X}}
\newcommand{\cY}{\mathscr{Y}}
\newcommand{\cZ}{\mathscr{Z}}

\newcommand{\rA}{\r{A}}
\newcommand{\rB}{\r{B}}
\newcommand{\rC}{\r{C}}
\newcommand{\rD}{\r{D}}
\newcommand{\rE}{\r{E}}
\newcommand{\rF}{\r{F}}
\newcommand{\rG}{\r{G}}
\newcommand{\rH}{\r{H}}
\newcommand{\rI}{\r{I}}
\newcommand{\rJ}{\r{J}}
\newcommand{\rK}{\r{K}}
\newcommand{\rL}{\r{L}}
\newcommand{\rM}{\r{M}}
\newcommand{\rN}{\r{N}}
\newcommand{\rO}{\r{O}}
\newcommand{\rP}{\r{P}}
\newcommand{\rQ}{\r{Q}}
\newcommand{\rR}{\r{R}}
\newcommand{\rS}{\r{S}}
\newcommand{\rT}{\r{T}}
\newcommand{\rU}{\r{U}}
\newcommand{\rV}{\r{V}}
\newcommand{\rW}{\r{W}}
\newcommand{\rX}{\r{X}}
\newcommand{\rY}{\r{Y}}
\newcommand{\rZ}{\r{Z}}

\newcommand{\C}{\bC}
\newcommand{\F}{\bF}
\newcommand{\N}{\bN}
\newcommand{\Q}{\bQ}
\newcommand{\R}{\bR}
\newcommand{\Z}{\bZ}

\newcommand{\Bdr}{\mathbb{B}_{\r{dR}}}
\newcommand{\Cp}{\mathbb{C}_p}
\newcommand{\Cl}{\mathbb{C}_{\ell}}
\newcommand{\Fp}{\mathbb{F}_p}
\newcommand{\Fl}{\mathbb{F}_{\ell}}
\newcommand{\Ga}{\mathbb{G}_{\r{a}}}
\newcommand{\Gm}{\mathbb{G}_{\r{m}}}
\newcommand{\Qp}{\mathbb{Q}_p}
\newcommand{\Ql}{\mathbb{Q}_{\ell}}
\newcommand{\Zp}{\mathbb{Z}_p}
\newcommand{\Zl}{\mathbb{Z}_{\ell}}

\newcommand{\Cpb}{\ol{\mathbb{C}}_p}
\newcommand{\Clb}{\ol{\mathbb{C}}_{\ell}}
\newcommand{\Fpb}{\ol{\mathbb{F}}_p}
\newcommand{\Flb}{\ol{\mathbb{F}}_{\ell}}
\newcommand{\hatZ}{\widehat{\mathbb{Z}}{}}
\newcommand{\Qpb}{\ol{\mathbb{Q}}_p}
\newcommand{\Qb}{\ol{\mathbb{Q}}}
\newcommand{\Qlb}{\ol{\mathbb{Q}}_{\ell}}
\newcommand{\Zpb}{\ol{\mathbb{Z}}_p}
\newcommand{\Zlb}{\ol{\mathbb{Z}}_{\ell}}

\newcommand{\ab}{\r{ab}}
\newcommand{\Ab}{\r{Ab}}
\newcommand{\an}{\r{an}}
\newcommand{\alg}{\r{alg}}
\newcommand{\Alg}{\r{Alg}}
\newcommand{\Ann}{\r{Ann}}
\newcommand{\Aut}{\r{Aut}}
\newcommand{\Ban}{\r{Ban}}
\newcommand{\ch}{\r{ch}}
\newcommand{\Cat}{\r{Cat}}
\newcommand{\Cod}{\r{Cod}}
\newcommand{\cont}{\r{cont}}
\newcommand{\Diag}{\r{Diag}}
\newcommand{\Div}{\r{Div}}
\newcommand{\Dom}{\r{Dom}}
\newcommand{\dR}{\r{dR}}
\newcommand{\End}{\r{End}}
\newcommand{\et}{\r{\'et}}
\newcommand{\Ext}{\r{Ext}}
\newcommand{\Field}{\r{Field}}
\newcommand{\Frac}{\r{Frac}}
\newcommand{\Frob}{\r{Frob}}
\newcommand{\Gal}{\r{Gal}}
\newcommand{\GL}{\r{GL}}
\newcommand{\Grp}{\r{Grp}}
\newcommand{\Hch}{\check{\r{H}}{}}
\newcommand{\Hdr}{\r{H}_{\r{dR}}}
\newcommand{\Het}{\r{H}_{\r{\'et}}}
\newcommand{\Hfet}{\r{H}_{\r{f\'et}}}
\newcommand{\Hom}{\r{Hom}}
\newcommand{\Hopf}{\r{Hopf}}
\newcommand{\id}{\r{id}}
\newcommand{\ind}{\r{ind}}
\newcommand{\Ind}{\r{Ind}}
\newcommand{\Isom}{\r{Isom}}
\newcommand{\Max}{\r{Max}}
\newcommand{\Mod}{\r{Mod}}
\newcommand{\Mon}{\r{Mon}}
\newcommand{\Mor}{\r{Mor}}
\newcommand{\Nr}{\r{Nr}}
\newcommand{\ob}{\r{ob}}
\newcommand{\op}{\r{op}}
\newcommand{\pro}{\r{pro}}
\newcommand{\Pro}{\r{Pro}}
\newcommand{\PSh}{\r{PSh}}
\newcommand{\Rep}{\r{Rep}}
\newcommand{\Res}{\r{Res}}
\newcommand{\Ring}{\r{Ring}}
\newcommand{\Sch}{\r{Sch}}
\newcommand{\Set}{\r{Set}}
\newcommand{\Sh}{\r{Sh}}
\newcommand{\SL}{\r{SL}}
\newcommand{\Spec}{\r{Spec}}
\newcommand{\Stab}{\r{Stab}}
\newcommand{\Sym}{\r{Sym}}
\newcommand{\Tor}{\r{Tor}}
\newcommand{\Tpl}{\r{Top}}
\newcommand{\tr}{\r{tr}}

\newcommand{\Cech}{$\check{\t{C}}$ech }
\newcommand{\Frechet}{Fr\'echet }
\newcommand{\Gelfand}{Gel'fand }
\newcommand{\Poincare}{Poincar\'e }
\newcommand{\Shnirelman}{Shnirel'man }
\newcommand{\Teichmuller}{Teichm\"uller }

\newcommand{\adm}{\r{adm}}
\newcommand{\cov}{\r{cov}}
\newcommand{\Cov}{\r{Cov}}
\newcommand{\Et}{\r{\'Et}}
\newcommand{\Fet}{\r{F\'et}}
\newcommand{\fet}{\r{f\'et}}
\newcommand{\fin}{\r{fin}}
\newcommand{\Ger}{\r{Ger}}
\newcommand{\PB}{\r{PB}}
\newcommand{\PLSp}{\r{PLSp}}
\newcommand{\pr}{\r{pr}}
\newcommand{\red}{\r{red}}
\newcommand{\Sub}{\r{Sub}}
\newcommand{\triv}{\r{triv}}
\newcommand{\whC}{\wh{\cC}}
\newcommand{\whH}{\wh{\r{H}}}
\newcommand{\WSet}{\r{WSet}}
\newcommand{\wtC}{\wt{\cC}}

\title{第3回Math-iine learning~Learning Functions: When Is Deep Better Than Shallow~}
\author{}
\date{1/28}


\begin{document}

% タイトルを出力
\maketitle
% 目次の表示
\tableofcontents

\section{Introduction}
\label{sec:Introduction}
この論文では,one-hidden layerのニューラルネットワークとdeep networkを比較する.

この論文で定理とされているものを記載する.
\begin{thm}
  Let $\sigma:\mathbb{R} \to \mathbb{R}$ be infinitely differentiable, and not a polynomial
  on any subinterval of $\mathbb{R}$.
  \begin{itemize}
    \item For $f \in W_{r,d}^{NN}$
    \begin{equation*}
     \mathrm{dist}(f,S_n) = \mathcal{O}(n^{-r/d})
    \end{equation*}
    \item For $f \in W_{H,r,2}^{NN}$
    \begin{equation*}
     \mathrm{dist}(f,D_n)= \mathcal{O}(n^{-2/d})
    \end{equation*}
  \end{itemize}
\end{thm}

\begin{thm}
There exists a constant $c > 0$ depending on $d$ alone with the following property.
Let  $\{ C_m \}$ be a sequence of finite subsets with $\{ C_m \} \subset [-cm,cm]^d$ with
\begin{equation*}
  1/m \preceq \mathrm{max}_{y \in K}\mathrm{min}_{x \in C}|x -y | \preceq  \eta(C_m)
\end{equation*}
 If $\gamma > 0$ and $f \in W_{\gamma,d}$
 then for integer $m \ge 1$ there exists
 $G \in N_{|Cm|,m}$ with centers at points in $C_m$
 such that
\begin{equation*}
  ||f -G||_d \preceq \frac{1}{m^{\gamma}}||f||_{\gamma,d}
\end{equation*}
 Moreover, the coefficients of $G$ can be chosen as linear combinations of the data $\{f(x) : x \in C_m\}$.
\end{thm}

\begin{thm}
  For each $v in V$, let  $\{C_{m,v}\}$ be a sequence of finite subsets as described in Theorem 2.
  Let $\gamma > 0$ and $f \in TW_{\gamma,2}$.
  Then for integer $m \ge 1$, there exists $G \in TN_{\mathrm{max} |C_{m,v}|m}(\mathbb{R}^2)$
  with centers of the constituent network $G_v$ at vertex $v$ at points in $C_{m,v}$ such that
\begin{equation*}
  ||f - G||_{\mathcal{T}} \preceq \frac{1}{m^{\gamma}}||f||_{\mathcal{T},\gamma,2}
\end{equation*}
Moreover, the coefficients of each constituent $G_v$
can be chosen as linear combinations of the data $\{f(x) : x \in C_{m,v} \}$.
\end{thm}

\begin{thm}
  \begin{itemize}
    \item[(a)] Let $\{C_m\}$ be a sequence of finite subsets of $\mathbb{R}^d$,
    such that for each integer $m \ge 1, C_m \subset C_{m+1},$
    $|C_m| \le c \mathrm{exp}exp(c_1m^2)$, and $\eta (C_m) \ge 1/m$.
    Further, let $f \in C_0(\mathbb{R}^d)$, and for each $m \ge 1$,
    let $G_m$ be a Gaussian network with centers among points in $C_m$,
    such that
    \begin{equation*}
      \mathrm{sup}_{m \ge 1 } m^{ \gamma }||f -G_m||_{\mathcal{T}} < \infty
    \end{equation*}
 Then $f \in W_{\gamma,d}$
 \item[(b)]
 For each $v \in V$ , let $\{ C_{m,v} \}$ be a sequence of finite subsets of $\mathbb{R}^{d(v)}$, satisfying the conditions as described in part
 (a) above. Let $f \in  \mathcal{T}C_0(\mathbb{R}^2), \gamma> 0$, and $\{ G_m \in \mathcal{T}N_{n,m} \}$ be a sequence where, for each $v \in V$,
 the centers of the constitutent networks $G_{m,v}$ are among points in $C_{m,v}$, and such that
\begin{equation*}
\mathrm{sup}_{m \ge 1}m^{\gamma}||f -G_m||_{\mathcal{T}} \ge \infty
\end{equation*}
 Then $f \in \mathcal{T}W_{\gamma,2}$.
\end{itemize}
\end{thm}

\section{Previous Work}
\label{sec:Previous Work}
以前の仕事自体には興味が無いので,自分が疑問に思う点をここに記載する.
全体の主張としては誤差が小さいものが存在するといってるだけ,誤差が$O(n^{-r/2})$まで落とせると言っている.

\begin{itemize}
  \item $\mathbb{Q},\mathbb{Q}_p$上でうまく定義できるか
  \item $n$や$d$の関係を明確にして,その状況で問題設定を解決したい.
  \item 計算量に関する考察は何かできないか
\end{itemize}

\section{Compositional functions}
\label{sec:Compositional functions}

\section{Main results}
\label{sec:Main results}
この章では,shallow network,deep networkの2つの場合に近似定理を述べる.
2つとは,ReLUによるdeep networkとdeep Gaussian networkである.
\it{degree of approximation}は以下で定義される.
\begin{equation}
 \mathrm{dist}(f,V_n)= \mathrm{inf}_{P \in V_n}||f - P||
\end{equation}
\begin{rem}
 $V_n$は関数の集合,実際にははニューラルネットワークとして定義される関数の集合として,使われていた.

\end{rem}
\subsection{Deep and shallow neural netwokrs}
\label{sub:Deep and shallow neural netwokrs}
$I^d:=[-1,1]^d,\mathbb{X}=C(I^d,\mathbb{R})$とし,
$||f||=\mathrm{max}_{x \in I^d}|f(x)|$とする.
$S_n$をn個のunitを持つshallow netoworkのなす集合とする.すなわち,
\begin{equation*}
 S_n:=\{f:\mathbb{R}^d \to \mathbb{R}|\mbox{ある}w_k ^in \mathbb{R}^d,b_k.a_k
 \in \mathbb{R}\mbox{が存在し,}f(x)=\sum_{k=1}^na_k \sigma(w_kx+b_k) \}
\end{equation*}
この時,訓練パラメータが$(d+2)n$個存在する.(メタ的で数学的ではない).
$W_{r,d}^{NN}$で$r$回連続偏微分可能であって,
$||f||+\sum_{1 \le |k|_1 \le r}||D^kf|| \le 1$を満たすもの全体とする.
また,$W_{H,r,2}^{NN}$を以下で定義する.
\begin{equation*}
  W_{H,r,2}^{NN}:=\{h | h=f_{11} \circ \cdots \circ f_{k2^k} (f_{ij} \in W_{r,2}^{NN})\}
\end{equation*}
$\mathcal{D}_n$を$S_n$に属する関数の合成で書けるもの全体とする.
上の書き方,かなりまずいけど, $f_11(f_21,f_22)$で表せるもの?つまり,dが実質2のものということですかね.
この時はパラメターの個数が$d=2^m$とした時に,$(d+2)m(1 + 2 + \cdots +2^{m-1})=(d+2)m(d-1)$となる.
\begin{thm}
 $\sigma :\mathbb{R} \to \mathbb{R}$を無限回微分可能であって,$\mathbb{R}$の任意の開区間上で,
 多項式でないとする. この時以下が成り立つ.
 \begin{enumerate}
   \item 任意の$f \in W_{r,d}^{NN}$に対し,
   \begin{equation}
     \mathrm{dist}(f,S_n)= \mathcal{O}(n^{-r/d})
   \end{equation}
   \item 任意の$f \in W_{H,r,d}^{NN}$に対し,
   \begin{equation}
     \mathrm{dist}(f,\mathcal{D}_n)=\mathcal{O}(n^{-r/2})
   \end{equation}
 \end{enumerate}
\end{thm}
\begin{proof}
  1つめの主張は他の論文にて示した.
  2つめの主張を示す.$f$が無限回微分可能な時,特にリプシッツ連続である.
  よって,$f(g_1,g_2)-f(P_1,P_2) \le M|g_1 - P_1||g_2 - P_2|$となる.
  これより,
  \begin{align*}
    |f(g_1,g_2)-P_0(P_1,P_2) |  & \le |f(g_1,g_2) - f(P_1,P_2)| +|  f(P_1,P_2) -P_0(P_1,P_2) | \\
    & \le M|g_1 - P_1||g_2 - P_2| +\mathrm{dist}(f,S_n)
  \end{align*}
となる.$|g_1 - P_1||g_2 - P_2|\le \mathcal{O}(n^{-r})$となるので.$f(g_1,g_2)-P_0(P_1,P_2) =\mathcal{O}(n^{-r/2})$となる.
これをinductiveに続けていけばよい.
\end{proof}
\begin{rem}
オーダとしてはこれが限界であることが示されている.
\end{rem}

\end{document}
