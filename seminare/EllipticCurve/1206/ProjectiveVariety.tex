%===============
%一行目に必ず必要
%文章の形式を定義
%===============
\documentclass{ujarticle}
%===============
%パッケージの定義、必要か不明
%===============
%この下4つを加えることで、mathbbが機能した
\usepackage{amsthm}
\usepackage{amsmath}
\usepackage{amssymb}
\usepackage{amsfonts}
%可換図式用パッケージ
\usepackage{amscd}
\usepackage[all]{xy}
\usepackage{tikz-cd}
%リンク用パッケージ
\usepackage[dvipdfmx]{hyperref}
%複数行コメント
%\usepackage{comment}

%タイトルデータ
\author{ari}
\title{Projectvie variety}
\date{2016/12/6}


%===============
%定理環境の設定
%セクション毎
%===============
\newtheorem{thm}{Theorem}[section]
\newtheorem{dfn}[thm]{Definition}
\newtheorem{prop}[thm]{Propostion}
\newtheorem{lem}[thm]{Lemma}
\newtheorem{cor}[thm]{Corllary}
\newtheorem{epl}[thm]{Example}
\newtheorem*{prob}{Problem}
\newtheorem*{rem}{Remark}
\newtheorem*{yodan}{余談,疑問}
\newtheorem{prf}{Proof}

\begin{document}

% タイトルを出力
\maketitle
% 目次の表示
\tableofcontents

射影代数多様体と射影スキームについて解説する.

\section{Projective Variety}
\label{sec:Projective Variety}

\subsection{Definition of Projecti Variety}
\label{sub:Definition of Projecti Variety}

射影空間と射影代数多様体について定義する.
\begin{dfn}
 $K$を体とする.射影空間とは…
\end{dfn}

\subsection{relation of affine and Projective}
\label{sub:relation of affine and Projective}
射影代数多様体とアフィン代数多様体に関係を述べる
特に非斉次化とその時の次元について触れる.

\section{Projective Scheme}
\label{sec:Projective Scheme}


\subsection{Graded A algebra}
\label{sec:Graded A algebra}

\subsection{Definition of  Projective Scheme}
\label{sub:Definition of Projective Scheme}

\section{Proper Morphism}
\label{sec:Proper Morphism}

\section{Projective Morphism}
\label{sec:Projective Morphism}


\end{document}
