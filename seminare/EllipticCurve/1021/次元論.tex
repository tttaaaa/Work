%===============
%一行目に必ず必要
%文章の形式を定義
%===============
\documentclass{ujarticle}
%===============
%パッケージの定義、必要か不明
%===============
%この下4つを加えることで、mathbbが機能した
\usepackage{amsthm}
\usepackage{amsmath}
\usepackage{amssymb}
\usepackage{amsfonts}
%可換図式用パッケージ
\usepackage{amscd}
\usepackage[all]{xy}
\usepackage{tikz-cd}
%リンク用パッケージ
\usepackage[dvipdfmx]{hyperref}
%複数行コメント
%\usepackage{comment}

%タイトルデータ
\title{Dimension}
\author{ari}
\date{2016/10/21}


%===============
%定理環境の設定
%セクション毎
%===============
\newtheorem{thm}{Theorem}[section]
\newtheorem{dfn}[thm]{Definition}
\newtheorem{prop}[thm]{Propostion}
\newtheorem{lem}[thm]{Lemma}
\newtheorem{cor}[thm]{Corllary}
\newtheorem{epl}[thm]{Example}
\newtheorem*{prob}{Problem}
\newtheorem*{rem}{Remark}
\newtheorem{prf}{Proof}

\begin{document}

% タイトルを出力
\maketitle
% 目次の表示
\tableofcontents

次元の定義をし,代数多様体での次元論を考える


\section{空間の次元}
\label{sec:空間の次元}

$X$を位相空間とする.$X$の次元を定義しよう.そのために必要な用語を定義する.
\begin{dfn}
 $X$が既約であるとは,$V_1,V_2$を$X$の閉集合で,$V_1 \cup V_2 =X$となる時,
 $V_1$か$V_2$が$X$と一致する.
 $V$が$X$の部分位相で既約となる閉集合であるとき, $V \subset X$が既約閉集合という.
\end{dfn}


$X_1 \subset X_2 \cdots  \subset X_r$を$X$の既約閉部分集合の真の包含列とする.
この時$r$を列の長さという.
$X$の次元$\mathrm{dim}X$を$X$の既約閉集合の列の長さの最大値とする.これは有限とは限らない.

\begin{prop}
 $Y \subset X$となるとき, $\mathrm{dim}Y \le \mathrm{dim}X$となる.
\end{prop}
\begin{proof}
 $V$を$Y$の既約閉集合とする時,$X$の閉包$\bar{V}$は既約閉集合となることを示せばよい.
 それは,$V_1 \subset V_2 \cdots \subset V_r$を$Y$の既約閉集合の列とすると,
 その閉包達による列は既約閉集合の列となる.また,閉包が一致している場合,元々も一致するので,これは真の列になる,
 これより,上の定理が成り立つ.
 $bar{V}$を既約閉集合となることを示す.$U_1 \cup U_2 \subset \bar{V}$とすると,$U_1 \cap Y$,
 $U_2 \cap V$は$V$の閉集合となるので,$V$の既約性より,どちらかは$V$に一致する.
 仮に$U_1 \cap V =V$とすると,$U_1 \supset V$となり,閉包の定義より,$U_1 =\bar{V}$となる.
\end{proof}



\begin{dfn}
 環$R$の素イデアルの列$\mathfrak{p}_1 \subset \mathfrak{p}_2 \subset \cdots \subset \mathfrak{p}_r$
 を素イデアルの列とする.$\mathfrak{p}$から始まる列の長さの最大値をheightという.
\end{dfn}
イメージは既約閉集合が閉部分多様体となるように定義している.
閉部分多様体は恐らく次元が落ちるので….

\end{document}
