%===============
%一行目に必ず必要
%文章の形式を定義
%===============
\documentclass{ujarticle}
%===============
%パッケージの定義、必要か不明
%===============
%この下4つを加えることで、mathbbが機能した
\usepackage{amsthm}
\usepackage{amsmath}
\usepackage{amssymb}
\usepackage{amsfonts}
%可換図式用パッケージ
\usepackage{amscd}
\usepackage[all]{xy}
\usepackage{tikz-cd}
%リンク用パッケージ
\usepackage[dvipdfmx]{hyperref}
%複数行コメント
%\usepackage{comment}

%タイトルデータ
\title{導来関手の存在証明}
\author{ari}
\date{2016/11/17}


%===============
%定理環境の設定
%セクション毎
%===============
\newtheorem{thm}{Theorem}[section]
\newtheorem{dfn}[thm]{Definition}
\newtheorem{prop}[thm]{Propostion}
\newtheorem{lem}[thm]{Lemma}
\newtheorem{ex}[thm]{Example}
\newtheorem*{prob}{Problem}
\newtheorem*{rem}{Remark}
\newtheorem{prf}{Proof}



\begin{document}
  \tableofcontents
\section{Introduction}
十分単射的対象を持つアーベル圏$\mathcal{A}$からアーベル圏$\mathcal{A}'$への左完全な加法的関手$\mathcal{F}$に対し,導来関手$R^i \mathcal{F}$がただひとつ存在する.
『コホモロジーのこころ』では,$\mathcal{A}$の対象$A$に対し,$R^i \mathcal{F}A$が定義されたが,射に関する部分が省略された.
そのため,射に関する部分を証明し,導来関手の存在を確認する.
つまり,$\mathcal{A}$の射$f:A \to B$に対し,
$R^iF(f):R^iF(A) \to F^iF(B)$が定まり,
以下を満たすことである.
\begin{eqnarray}
  R^iF(id) & = & id \\
  R^iF(f) \circ R^iF(g) &=& R^iF(f \circ g)
\end{eqnarray}

これは$f:A \to B$に対し以下を示せばよい.

\begin{prop}
 $\mathcal{A}$の射$f:A \to B$と$A,B$の単射的分解$I^i,J^i$に対し,以下を可換にするような射$f^i:I^i \to J^i$が存在する.

 \xymatrix{
  0 \ar[r] & A \ar[d]^f  \ar[r]^{\alpha^{-1}}  \ar@{}[rd]|{\circlearrowright}  & I^0 \ar[r]^{\alpha^{0}}  \ar[d]^{f^0}  \ar@{}[rd]|{\circlearrowright} & I^1 \ar[d]^{f^1}  \ar@{.>}[r]  & \\
 0 \ar[r]  & B               \ar[r]^{\beta^{-1}}   & J^0 \ar[r]^{\beta^{0}} & J^1   \ar@{.>}[r]  &}
\end{prop}

\begin{prop}
  $\mathcal{A}$の射$f:A \to B$と$A,B$の単射的分解$I^i,J^i$に対し,上の命題を満たす,$f^i,g^i$はホモトピックとなる.
\end{prop}
この2つの命題を示せば関手となることがいえる.軽く確認しておくと,
ホモトピックであれば,$\mathcal{F}(f^i)=\mathcal{F}(g^i)$となるので,$f$のみで,$f^i$の取り方によらなない.また,そのような$f^i$が必ず存在するので$R^if$がwell-definedである.
また,$R^i \mathcal{F}(id)=id $と,$R^i\mathcal{F}(f) \circ R^i \mathcal{F}(g) = R^iF(f \circ g)$も射の取り方によらないことと$\mathcal{F}$の関手性から示せる.

\section{Proof of proposition 1}
\label{sec:Proof of proposition 1}
1つめの命題を示す.$f^0$と$f^1$の存在を示す.
$f^2$以降は$f^1$の場合と同じ議論より言える.
\begin{itemize}
  \item 0の時
\end{itemize}

\xymatrix{
 0 \ar[r] & A \ar[d]^f  \ar[r]^{\alpha^{-1}}  \ar@{}[rd]|{\circlearrowright}  & I^0  \ar@{.>}[d] \\
0 \ar[r]  & B               \ar[r]^{\beta^{-1}}   & J^0 }

上の図の斜線の射の存在を示す.これは$\alpha^{-1}:A \to I^0$が単射で,
$\beta^{-1} \circ f : A \to J^0$となるので,$J^0$が単射的対象であることから,
斜線の射の存在が従う.

\begin{itemize}
  \item 1の時
\end{itemize}

\xymatrix{
 A \ar[d]^f \ar[r] & I^0 \ar[d]^{f^0}  \ar[r]^{}    &  \mathrm{Im}\alpha^0 \simeq I^0/\mathrm{Ker}{\alpha^0} \simeq I^0/\mathrm{Im}{\alpha^{-1}}
 \ar[r]   \ar@{.>}[ld]^{\tilde{f^0}}& I^1  \ar@{.>}[d]  \\
B \ar[r]  & J^0  \ar[r]^{} &    \mathrm{Im}\beta^0 \simeq J^0/\mathrm{Ker}{\beta^0} \simeq J^0/\mathrm{Im}{\beta^{-1}}         \ar[r]^{}   & J^1 }

$\tilde{f^0}$ の存在を示す.$I^0/\mathrm{Im}\alpha^{-1} \to J^0$が定義できれることを示す.それは$\beta^{0} \circ f^{0} \circ \alpha^{-1}(A) $が図式の可換性から,$ \beta^{0} \circ \beta^{-1} \circ f(A)$と等しいことと,
$ \beta^{0} \circ \beta^{-1} =0$より言える.これから$\mathrm{Im}\alpha^0 \to J^0 \to J^1$という射が定義でき,$\mathrm{Im}\alpha^0 \to I^1$が単射であり,$J^1$が単射的対象であることから,$I^1 \to J^1$で上の図式を可換にするものが存在する.

\section{proof of Propostion 2}
\label{sec:proof of Propostion 2}
すいません,時間がなかったので,また今度….
証明自体は
『コホモロジーのこころ』67Pと同様にすればよい.(ここでいう$h^i$として$f^i - g^i$を取れば良い.)

\end{document}
