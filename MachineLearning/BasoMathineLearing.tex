%===============
%一行目に必ず必要
%文章の形式を定義
%===============
\documentclass{ujarticle}
%===============
%パッケージの定義、必要か不明
%===============
%この下4つを加えることで、mathbbが機能した
\usepackage{amsthm}
\usepackage{amsmath}
\usepackage{amssymb}
\usepackage{amsfonts}
%可換図式用パッケージ
\usepackage{amscd}
\usepackage[all]{xy}
\usepackage{tikz-cd}
%リンク用パッケージ
\usepackage[dvipdfmx]{hyperref}
%複数行コメント
%\usepackage{comment}


%===============
%定理環境の設定
%セクション毎
%===============
\newtheorem{thm}{Theorem}[section]
\newtheorem{dfn}[thm]{Definition}
\newtheorem{prop}[thm]{Propostion}
\newtheorem{lem}[thm]{Lemma}
\newtheorem{ex}[thm]{Example}
\newtheorem*{prob}{Problem}
\newtheorem*{rem}{Remark}
\newtheorem{prf}{Proof}

\begin{document}

\section{Introduction}
\label{sec:Introduction}
機械学習,及び統計の基礎について自分が気になる点を中心に記述する.

\begin{itemize}
  \item t-SNEのアルゴリズム,及び実装
  \item ベイズによる最適化
  \item 確率論の基本
  \item ガウス分布,分散,フィッシャー情報行列
  \item 機械学習の基本
  \item NNの基本
  \item TDE
  \item kernel法
\end{itemize}


\section{t-SNE}
\label{sec:t-SNE}
\subsection{定義}
\label{sub:定義}
距離に基づく確率分布

\subsection{実行結果}
\label{sub:実行結果}

\section{NNの基本}
\label{sec:NNの基本}



\end{document}
