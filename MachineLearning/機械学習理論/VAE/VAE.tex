%===============

%一行目に必ず必要
%文章の形式を定義
%===============

\documentclass{ujarticle}


%===============ややゆ
%パッケージ定義、必要か不明
%===============分からないけど、。
%この下4つを加えることで、mathbbが機能した
\usepackage{amsthm}
\usepackage{amsmath}
\usepackage{amsfonts}
\usepackage{amssymb}
\usepackage{my-default}
%リンク用パッケージ
\usepackage[dvipdfmx]{hyperref}
%tikz用パッケージ
\usepackage[dvipdfmx]{graphicx}
\usepackage{tikz}
\usepackage{tikz-cd}
\usepackage{my-default}
\usepackage{pdfpages}


%===============
%定理環境の設定
%セクション毎
%===============


\begin{document}

\section{Introduction}
\label{sec:Introduction}
VAEについて説明する.

\subsection{データの数学化}
\label{sub:データの数学化}
現実のデータを数学的に扱うためには,現実に対して仮説をおくこと(モデル化)になる.
VAEはベイズ統計の枠組みで考えているため,ベイズ統計で共通の以下の前提を考えている.
\begin{itemize}
  \item 真の分布と呼ばれる確率分布が存在し,入力データはその確率分布に従う.
  \item 予め事前分布として確率分布$p_{\theta}(z)$を定めておく.
  \item $n$回試行し得られた入力データ$D=\{x_1,\dots,x_n\}$に対し,$p_{\theta}(D|z)$が最大となる確率分布$p_{\theta}(x)$を真の分布とみなす.
\end{itemize}
画像を集めてきて学習させている時,画像が確率分布であるという感覚はないかもしれない.
だが,ベイズ統計の考え方では,例えば,赤色の確率が$1/3$青色の確率が$1/3$,白色の確率が$1/3$の確率分布
として考えているのである.
この考え方が正しいかはさておき,ひとまずそう設定することで具体的に答えの出せる問題に落とし込んである.


\subsection{VAEの考え方}
\label{sub:VAEの考え方}
VAEではベイズ統計の考え方に基づき$p_{\theta}(X)$を求めたいのだが,実際に$p_{\theta}(D|z)$が最大となる$p_{\theta}(x)$を求めることは特別な場合しかできない.
大抵の場合は解析的に計算できない積分を求める必要がある.

ベイズ統計では,この困難な積分に対応するため,いくつかの方法が考えられてきた.
一つがMCMC等の乱数によって近似値を計算すること.
別の方法として,$p_{\theta}(D|X)$にできるだけ近く,かつ計算可能な$q_{\phi}(D|X)$を定める方法である.

VAEで出てくる変分ベイズは後者の考えた方を使う.
こうした考え方は変分推論と呼ばれる.
「"近い"」は$KL(q(D|X)|p(D|X))$が最小になると定める.(KLが妥当なのか,また,pとqの入れ替えが妥当なのかは不明.)


http://nzw0301.github.io/notes/vae.pdf
よりわかりやすい説明ないからどうしようかな.

\subsection{ニューラルネットワークでの計算}
\label{sub:ニューラルネットワークでの計算}
なぜ,これでいいのかがわからなくなってしまった.
X,Zに対し,適切に値を出力する.(この構成でいいのかは不明)
ここで$p_{\theta}(X)$等の意味を大きく以前と変えてしまっている気がする.
autoencoderの形にして学習させた.
実際に生成させる時にデータを入力させないので,
ただ確率分布を求めるだけでなく,その確率分布を実現させる乱数生成器を作る必要があった.
そのために事前分布p(z)に基づき正規分布に従う乱数を与えたら,画像を生成する仕組みが欲しかった.
そのためにAutoEncoderの形にした.


\end{document}
