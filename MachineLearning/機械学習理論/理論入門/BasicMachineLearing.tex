%=============== 

%一行目に必ず必要
%文章の形式を定義
%===============わ

\documentclass{ujarticle}


%===============ややゆ
%パッケージ定義、必要か不明
%===============分からないけど、。
%この下4つを加えることで、mathbbが機能した
\usepackage{amsthm}
\usepackage{amsmath}
\usepackage{amssymb}
\usepackage{amsfonts}
\usepackage{my-default}
%リンク用パッケージ
\usepackage[dvipdfmx]{hyperref}
%tikz用パッケージ
\usepackage[dvipdfmx]{graphicx}
\usepackage{tikz}
\usepackage{tikz-cd}
\usepackage{my-default}
\usepackage{pdfpages}


%===============
%定理環境の設定
%セクション毎
%===============


\begin{document}

\section{Introduction}
\label{sec:Introduction}
機械学習,及び統計の理論的な面について解説する.

\begin{itemize}
  \item 機械学習の基本
  \item t-SNEのアルゴリズム,及び実装
  \item ベイズによる最適化
  \item 確率論の基本
  \item ガウス分布,分散,フィッシャー情報行列
  \item NNの基本
  \item TDE
  \item kernel法
  \item CTC
\end{itemize}
入力しにくいが不可能ではない.


\part{機械学習入門}
この章では機械学習の定義と実際に使われる手法の例について記載する.

機械学習とは,機械をデータを用いて訓練させて,訓練以外のデータに対しても適切に応答を返す仕組み全般に対する総称である.
昔は機械には条件を明記し,手続きを記述していた.
だが,5000兆通りの方法を全て記述していたらいくら時間があっても足りない.
つまり,全ての条件や全てのデータを入手することは現実的に不可能であるという認識が機械学習を考える背景となっている.
実際に現在の機械学習では,以下のプロセスによって機械学習をするシステムを実現している.
\begin{enumerate}
  \item 現象レベルでの問題設定
  \item 入力,出力データの決定
  \item 機械学習用モデル(判断基準)の設定
  \item モデルを数学的に解決
  \item 結果の検証
\end{enumerate}
これらを詳しく見る.

\paragraph{現象レベルでの問題設定}
自然現象のレベル間で問題を設定する.
例えば,猫か否か分類したい.適切に会話したい等.
ただ,このレベルでは機械へのインターフェースではないため,機械の入力へ落とし込む必要がある.

\paragraph{入力,出力データの決定}
機械へのインターフェースを設定する.

\paragraph{機械学習用モデル(判断基準)の設定}

\paragraph{モデルを数学的に解決}

\paragraph{結果の検証}

\subsection{教師あり学習}
\label{sub:教師あり学習}
教師あり学習とは入力データに,正解を含める手法のことを指す.
この場合モデルとしては
$f : \mathbb{R}^d \to \mathbb{R}$を
$f|_{\mathcal{D}}$がわかっている前提で最も"よい"関数$f$を求める.

その判断方法として様々なものがある.

\subsection{線形回帰モデル}
\label{sub:線形回帰モデル}
この場合二乗誤差が最小となる1次関数を求める.

結果について記述する.
教師あり学習の基本的な手法についてまとめる。
教師あり学習とは入力データと正解ラベルを用いて入力データに対し、もっとも正解ラベルを出力してくれるモデルを構成する手法である。
数学的に定義する.
$x_1,\dots,x_m$を$a_i \in \mathbb{R}^m$とし,
$y_1, \dots,x_m$を$x_i \in \mathbb{R}$とする.

機械学習で,数理モデルを作る時,モデルに出てこないパラメータは暗に不変だと推測している.


\subsection{交差検証}
\label{sub:交差検証}

交差検証(Cross-validation)とは、統計学において標本データを分割し、その一部をまず解析して、
残る部分でその解析のテストを行い、
解析自身の妥当性の検証・確認に当てる手法を指す。データの解析
(および導出された推定・統計的予測)がどれだけ本当に母集団に対処できるかを良い近似で検証・確認するための手法である。


\part{Bayes}
この解説は特に「ベイズ統計の理論と方法」を読んだ疑問に感じた事調べたことを中心に記述する.
\section{Bayes統計}
\label{sec:Bayes統計}


事前に以下の2つを仮定する.
\begin{itemize}
  \item 現実は「真の分布」と呼ばれる確率分布によって記述(モデル化)できる
  \item ただし,我々は現象しか観測できないため,真の分布を知ることはできない.
\end{itemize}
これが正しいと考えた場合,観測した現象を基に真の分布を推測できる仮定し,推測する方法を研究する分野が
統計的学習理論,あるいは統計的推測等といわれる.
統計的学習理論にも様々な推測方法があるが,Bayes統計では以下を用いて真の分布を推測する.
\begin{itemize}
  \item 事前分布
  \item 確率モデル
\end{itemize}

最初に数学的にベイズ統計を定式化する.
$x_i \in \mathbb{R}^n$の列$\{ x_i \}_{i= 1,\dots ,d}$に対し,
ある$\mathbb{R}^{nd}$上の確率測度$\mu$と$\mathbb{R}^{nd}\times [0,1]$上の確率測度$\mu_{\theta}$
が与えられたとする.ここで,対応する確率分布関数を,それぞれ$p(x),p(x,\theta)$とする.
この時
この時,$\{ x_i \}_{i= 1,\dots d}$とが起きる確率が最大となる確率


\begin{rem}
この辺は定義が曖昧な部分もあり,ある人は何がベイズで何が頻度かはかなり人によるので(少なくとも互いにcomplementではない),
「事後分布が出てきたらベイズ」と思っておくのがケンカになりにくいとまで言っていた.
\end{rem}


統計学の考え方,及びBayes統計とは何かを明確にし,Bayes統計の基本的な考え方を明示する.
私は統計学とは「数学」を使って現実を分析あるいは予測する学問だと考えている.
一般的にはデータに基づき現実を分析,予測すると考えてもいいかもしれない.
だが,ここでは数学と敢えて記述する.それはデータには不正確さがあるのに対し,数学には不正確さがないためである.
統計は現実の問題であり,ある意味「明確な」答えはない.
だが,どこに問題があるかは明確にできるはずだと考えている.数学面には不明確な部分はない.
例えばデータをモデル化する時の近似が問題なのか,モデル化したものを具体的に解く数学的手法が問題なのか,
そもそも入力データが問題なのか等課題を明確にできるのではないかと予測している.

ではBayes統計とは何かというと.
現実に次起こるものを確立分布として表現し.
その確率分布を事前分布を基に入力されたデータによって逐次改善して真の確率分布を得る方法のことである.
もう少し具体的に説明しよう.


頻度とベイズについて説明する。

\section{t-SNE}
\label{sec:t-SNE}
\subsection{定義}
\label{sub:定義}
距離に基づく確率分布である.

\subsection{実行結果}
\label{sub:実行結果}

\section{NNの基本}
\label{sec:NNの基本}

NNの基本的なアルゴリズムについて述べる.
NNの定義をする.


\section{CTC}
\label{sec:CTC}
CTC(CONNECTIONIST TEMPORAL CLASSIFICATION)について説明する.

時系列のニューラルネットワーク学習に使われている.
例えば時系列に対し,単語を出力させるアルゴリズムとして使われる.

入力データを$B \in \mathbb{R}^n$とすする.出力は確率となる.
具体的には有限集合$A$と$a \in A$に対し,$0 \le p_a \le 1$で$\sum p_a =1$となるものが出力となる関数である.
今時刻$t$に$k$が出力された確率を$y_k^t$と表す.
このとき時系列データ,例えば文字列$\pi="abc"$が出力される確率を
\begin{equation*}
 p(\pi \mid x ) = \Prod y_{\pi_t}^t
\end{equation*}
と計算する。
ただ、音声の場合は文字が続いているもの、あるいは聞き間違いなどを補正するために、
確率分布を変更する。
例えば$abc $や$aabc$と判定された場合はすべて$abc$とみなすようにする。
数式的に定式化しづらいが、そうした変換を仮に$F$と書き、(定義域の変更に対応できない)以下のように定義する.

\begin{equation*}
  p( \el \mid x) = \sum_{ \pi \in F^{-1}(\el)} p(\pi \mid x )
\end{equation*}
こちらの確率が最大となるものを求める正解として与える.

逆像を具体的に計算するために、漸化式を定義して計算する.
前の状態が前が空白か前が別の文字の場合か前も同じ文字かしかないので、そのつもりで計算すればよい.
また、時刻とともに位置(文字数+空白部分)があるので、その文字数も含めて漸化式にする.



\end{document}
