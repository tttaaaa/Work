%===============
%一行目に必ず必要
%文章の形式を定義
%===============
\documentclass{ujarticle}
%===============
%パッケージの定義,必要か不明
%===============
%この下4つを加えることで,mathbbが機能した
\usepackage{amsthm}
\usepackage{amsmath}
\usepackage{amssymb}
\usepackage{amsfonts}
%可換図式用パッケージ
\usepackage{amscd}
\usepackage[all]{xy}
\usepackage{tikz-cd}
%リンク用パッケージ
\usepackage[dvipdfmx]{hyperref}
%複数行コメント
%\usepackage{comment}


%タイトルデータ
\title{TESTING THE MANIFOLD HYPOTHESIS}
\author{test}
\date{2017/01/29}
%===============
%定理環境の設定
%セクション毎
%===============
\newtheorem{thm}{Theorem}[section]
\newtheorem{dfn}[thm]{Definition}
\newtheorem{prop}[thm]{Propostion}
\newtheorem{lem}[thm]{Lemma}
\newtheorem{cor}[thm]{Corllary}
\newtheorem{epl}[thm]{Example}
\newtheorem*{prob}{Problem}
\newtheorem*{rem}{Remark}
\newtheorem{prf}{Proof}
\newtheorem{clm}{Claim}
\newtheorem{hyp}{HYPOTHESIS}
\newtheorem{mtm}{Main Thoerem}

%この論文紹介用定義
\newcommand{\bh}[2]{B_{\mathcal{H}}(#1,#2)}
\newcommand{\bpa}{B_{\Pi_1}(0,r)}
\newcommand{\bpb}{B_{\Pi_2}(0,r)}
\newcommand{\bpd}{B_{\Pi}(0,r)}
\newcommand{\bp}[3]{B_{\Pi_{#3}}(#1,#2)}
\newcommand{\gn}[4]{||\Gamma_{#1}||_{C^{1,1}(\bp{#2}{#3}{#4})}}
\newcommand{\gnaaad}{||\Gamma_1||_{C^{1,1}(\bp{z_1}{r_1}{})}}
\newcommand{\gnd}{||\Gamma||_{C^{1,1}(\bpd)}}
\newcommand{\gdvt}{\mathcal{G}(d,V,\tau)}
\newcommand{\Me}{M_{erm}}
\newcommand{\Px}{\Pi_x}
\newcommand{\Py}{\Pi_y}


\begin{document}

\part{1}
\label{part:1}



\section{Introduction}
\label{introduction}


\subsection{Abstract}
\label{sub:Abstract}

最初に論文に記載されている内容を説明する.
数学的に示していることをざっくりとした(数学的には不正確な)日本語で説明すると以下の二点である.
\begin{itemize}
  \item 多様体仮説がある多様体に対して,成り立つと仮定する.この時,
  実際に多様体仮説が成り立つ多様体$M$を数学的に構成した.
  \item 上の多様体$M$を具体的に求めるアルゴリズムを構成した.
\end{itemize}

(この論文)では,入力となるデータに関する性質を記述できていないので,
多様体仮説が成り立つことは確認できない.
だが,多様体仮説が成り立つかはわからないが,多様体仮説が成り立つならば,
求めたい多様体(といっていいもの)が定義でき,さらにそれを実際に求めるアルゴリズムを構成できたので,
実際に実験してみればよいと主張している.
もし数学的に多様体仮説を示すのであれば,入力データの偏りに相当する仮定が必要になる.
また,多様体仮説を聞いて,多様体の形に基づいた,クラスタリングが定義できるのではないかと予想していたが,
この論文では特に記述はなかった.
(例えば,連結成分等でうまく定義できて欲しいものだが...)

結果を具体的に記述するため,最低限の数学的な用語を定義する.
\subsection{Notation and Definition}
\label{sub:Notation and Definition}

\begin{dfn}
  有限次元とは限らない$\mathcal{H}$を$\mathbb{R}$-ベクトル空間が
  以下を満たす時,\textbf{ヒルベルト空間}という.
  \begin{itemize}
    \item 内積が定義できる.すなわち,以下を満たす写像$< \cdot , \cdot >: \mathcal{H} \times \mathcal{H} \to \mathbb{R}$が存在する.
      \begin{enumerate}
        \item 任意の$x_1,x_2,y \in \mathcal{H}$に対し,${<}x_1 +x_2,y{>}= {<}x_1,y{>} +{<}x_2,y{>}$
        \item 任意の$x,y \in \mathcal{H},a \in \mathbb{R}$に対し,$<ax,y> =a<x,y>$
        \item 任意の$x,y \in \mathcal{H}$に対し,$<x,y>=<y,x>$
        \item 任意の$x \in \mathcal{H}$に対し,${<}x,x{>} \ge 0$であり,${<}x,x{>}=0$と$x=0$は同値である.
      \end{enumerate}
    \item 内積が定める距離について完備である.
  \end{itemize}
\end{dfn}
\begin{rem}
上で定めた内積のことを正定値対象双線型形式という.
\end{rem}
\begin{rem}
  $x,y \in \mathcal{H}$に対し,$d(x,y):=\sqrt{{<}x-y,x-y{>}}$
 とすると距離の公理を満たす.
 距離の公理とは以下の3つのこと.
 \begin{enumerate}
   \item $d(x,y)=d(y,x)$
   \item $d(x,y)=0$は$x=y$と同値
   \item $d(x,y)+d(y,z) \ge d(x,z)$
 \end{enumerate}
\end{rem}
\begin{rem}
 距離空間$X$が完備とは,任意のコーシー列$\{x_n\}_{n \in \mathbb{Z}_{\ge 0}}$が,ある$x \in X$に対する収束列となること.
\end{rem}
\begin{epl}
 $\mathbb{R}^n$は標準内積により,ヒルベルト空間になる.
\end{epl}

この論文で使われる"多様体"の定義をする.
\begin{dfn}[reach]
  $M$を$\mathcal{H}$の部分集合とし.$x\in \mathcal{H}$に対し,$d(x,M)$を$\mathrm{inf}_{y \in M}d(x,y)$で定める.
  実数$\tau$が$M$の\textbf{reach}とは,$d(x,M)< \tau$となる任意の$x$に対し,
  $y\in M$が存在し,$y$と異なる任意の$z\in M$に対し,$d(x,z) > d(x,M)=d(x,y)$となること.
\end{dfn}
\begin{epl}
 半径$r$の円周$\{(x,y) \in \mathbb{R}^2 \mid x^2+y^2=r^2 \}$のreachは$r$になる.
\end{epl}
\begin{rem}
 reachの定義は初めて見た.恐らく多様体をクラスタリングするために,定義されているのだと思う.
\end{rem}
\begin{dfn}
 位相空間$X$の部分集合$Y$が\textbf{稠密}とは$Y$の閉包が$X$に一致することをいう.
 位相空間$M$が稠密な加算部分集合を持つ時,\textbf{可分}という.
\end{dfn}
\begin{epl}
 実数体$\mathbb{R}$に通常のユークリッド位相を定めた時,有理数体$\mathbb{Q}$は稠密になる.
 また$\mathbb{Q}$は加算集合のため,$\mathbb{R}$は可分である.
\end{epl}

\begin{dfn}[Tangent Space]
$H$を可分ヒルベルト空間とする.閉集合$A \subset \mathcal{H}$と$a \in A$に対し,$Tan^0(a,A)$を以下で定める.
\begin{equation*}
  \{ v \in \mathcal{H} \mid \mbox{任意の} \epsilon \mbox{に対し,ある} b \in A \mbox{が存在し,}
  0 <| b- a| < \epsilon \mbox{に対し} |\frac{v}{|v|} - \frac{b-a}{|b-a|} < \epsilon.\}
\end{equation*}
$Tan(a,A)$を$\{x \in H| x -a \in Tan^0(a,A)\}$で定め,$a$での\textbf{Tangent Space}という.
\end{dfn}

\begin{prop} $A$を $\mathbb{R}^n$の閉部分集合とする.この時
  \begin{equation*}
   reach(A)^{-1}=\mathrm{sup}\{2|b-a|^{-2}d(b,Tan(a,A)) | a.b \in A\}
  \end{equation*}
\end{prop}
\begin{dfn}
 $V \subset \mathcal{H}$が$d$次元\textbf{アフィン空間}とは,ある$a \in \mathcal{H}$が存在し,$\{ x \in \mathcal{H} \mid x + a \in V \}$が$d$次元のベクトル空間になること.
\end{dfn}
\begin{dfn}[$C^r$-submfd]
ヒルベルト空間$\mathcal{H}$の閉集合$M$は以下を満たす時,
$d$次元$C^r$\textbf{級部分多様体}という.
\begin{itemize}
  \item 任意の$p \in M$に対し,$x \in U$となる$\mathcal{H}$の開部分集合と$C^r$級写像$\phi:U \to \mathcal{H}$が存在する.
  \item $\phi|_{U \cap M}$は終域を像に制限すると微分同相写像である
  \item $d$次元アフィン空間の族$\{ V_i \}_{i \in I}$に対し,$\phi(U \cap M)=\cap_{i \in I} V_i \cap \phi(U)$が成り立つ.
\end{itemize}
  $B_{\mathcal{H}}:= \{ x \in \mathcal{H} \mid d(x,x) \le 1 \}$とする.
$\gdvt$を体積$V$以下であり,reachが$\tau$以上となる$B_{\mathcal{H}}$に含まれる$d$次元(境界のない)$C^r$級多様体全体のなす集合とする.

\end{dfn}
\begin{rem}
 多様体は普通の数学書では上記のようには定義されない.
 $\mathcal{H}$を固定して考えたいがために,上記のように定義したのだと思われる.
\end{rem}
\begin{rem}
 この論文では,後でもう一つ多様体と呼ぶものが出てくる.$C^r$級部分多様体が必ずしも後で出てくる多様体とは限らないことに注意せよ.
\end{rem}

\begin{dfn}
 $\Omega$を集合とする.
 $F$を$\Omega$上の$\sigma$-加法族,
 $\mu_P$を$F$上の非負測度で,$\mu_P(\Omega)=1$とする.
 この時,$(\Omega,F,\mu_P)$を確率空間という.
\end{dfn}
\begin{dfn}
 $X:\Omega \to [-\infty,\infty]$が確率変数とは任意の$a \in [-\infty,\infty]$で,$\{x \in \Omega \mid X(x) \le a \} \in F$となること.
\end{dfn}
\begin{dfn}
 $X$を確率変数とした時,$P(x) = \mu_P(X <x)$と書ける時,$P$を確率分布関数という.
\end{dfn}


\subsection{Manifold Hypothesis}
\label{sub:多様体仮説}
この論文で考える多様体仮説とそのために必要な用語を定義する.
可分なヒルベルト空間を$\mathcal{H}$,その単位球を$B_{\mathcal{H}}$,
$B_{\mathcal{H}}$上の確率分布関数を$P$とする.
\begin{equation*}
 \mathcal{L}(M,P):=\int_{x \in B_{\mathcal{H}}} d(x,M)^2\frac{dP(x)}{dx}dx.
\end{equation*}
とする.
確率分布関数$P$によるi.i.d(独立試行)が行われた時,以下が成り立つことを多様体仮説と呼ぶ.
\begin{hyp}
  ある$M \in \mathcal{G}(d,CV,\tau/C)$が存在し,$\mathcal{L}(M,P) \le C \epsilon$となる.
\end{hyp}
\begin{rem}
 論文中でこれを多様体仮説と直接呼んではいないが,内容から判断して多様体仮説と呼ぶことにした.
\end{rem}
\begin{rem}
 この時点でもわかるように,$d,C,V,\tau$としてどのような値を取るべきかは不明.
\end{rem}

\subsection{Main Theorem 1}
\label{sub:主結果その1}

この論文の1つめの主定理を述べる.
\begin{mtm}
\label{mth1}
 ある$r > 0$に対し,
 \begin{equation*}
  U_{\mathcal{G}(1/r)}:=CV( \frac{ 1 }{ \tau^d } + \frac{ 1 }{ (\tau r)^{d/2} } )
 \end{equation*}
 とする.また,
 \begin{equation*}
  s_{\mathcal{G}}(\epsilon,\delta):=C(\frac{ U_{\mathcal{G}}(1/\epsilon) }{ \epsilon^2 }(\mathrm{log}^4
  (\frac{U_{\mathcal{G}}(1/\epsilon)}{\epsilon})) + \frac{ 1 }{ \epsilon^2 }\mathrm{log}\frac{ 1 }{ \delta }  )
 \end{equation*}
とする.
$s \ge s_{\mathcal{G}}(\epsilon,\delta)$とし,$x=\{ x_1,\dots,x_s\}$を確率分布$P$による独立試行により,得られた集合とする.
$P_X$を$X$上の一様確率分布関数とする.
$M_{erm}(x) \in \mathcal{G}(d,V,\tau)$を
\begin{equation*}
 L(M_{erm}(x),P_X) - \mathrm{inf}_{M \in \mathcal{G}(d,V,\tau)} L(M,P_X) < \frac{ \epsilon }{ 2 }
\end{equation*}
を満たす$\mathcal{G}(d,V,\tau)$で,以下を最小にするものとする.
\begin{equation*}
 \sum_{i=1}^sd(x_i,M)^2
\end{equation*}
この時,
\begin{equation*}
 \mathbb{P}[L(M_{erm}(x),P) - \mathrm{inf}_{M \in \mathcal{G}(d,V,\tau)} L(M,P) < \epsilon] > 1- \delta
\end{equation*}
となる.
\end{mtm}
\begin{rem}
 $s$次元の直積測度で考えた時,上の条件を満たす$x$全体の集合をその測度に代入すると$1- \delta$より大きいという意味.
\end{rem}
\begin{rem}
 この定理が何を示しているかというと,$L(M,P)$の極小値に十分近い値を取る,多様体$M_{erm}(x)$を構成できることが言えた.$M_{erm}(x)$は最適化問題の解として得られるので,
 極小値よりもかなり具体的になっている.
 特に,$M_{erm}(x)$が確率を計算しなてくも求められるのがよい.基本的に$P_X$は未知であることを前提に問題と解くので,$P_X$に依存しないのは強い結果である.
\end{rem}
\begin{rem}
  仮定からわかるように,$\epsilon.\delta$の値にも強く依存する.私はまだ,これらとして何を取るべきかはわかっていない.計算してみて,どのぐらい大きい$s$が必要か推測(計算)したい.
\end{rem}

\section{Preparation for Main Theorem 1}
\label{Preparation for Main Theorem 1}
主定理\ref{mth1}を示すための準備を行う.
多様体を考察することで以下を示す.
\begin{clm}
  $M \in \gdvt$とし,$\Pi_x$を$\mathcal{H} \to \mathrm{Tan}(x,M)$への射影とする.
  十分大きい(controlされた?)定数$C$に対し,
  \begin{equation*}
   U:= \{ y \mid |y - \Pi_xy| \le \tau/C \} \cap \{ y \mid |x - \Pi_x y| \le \tau/C \}
  \end{equation*}
  とする.この時,$C^{1,1}$級関数$F_{x,U}:\Pi_x(U) \to \Pi_x^{-1}(\Pi_x(0))$で以下を満たすものが存在する.
  \begin{enumerate}
    \item $F_{x,U}$のLipschitz constant of the gradientが$C$以下である.
    \item $ \displaystyle
    M \cap U = \{ y + F_{x,U}(y) \mid y \in \Pi_x(U) \} $
  \end{enumerate}
\end{clm}

\subsection{Proof of Claim 1}
\label{sec:Proof of CLAIM 1}

$\mathcal{H}$を(無限次元でもよい)ヒルベルト空間とする.
$D$-planeを$\mathcal{H}$の$D$次元部分ベクトル空間とする.
$DPL$で$D$-plane全体のなす空間を表す.
$\Pi,\Pi^\prime \in DPL$に対し,$T: \mathcal{H} \to \mathcal{H}$を直交変換(内積が不変な変換)
であって,$T(\Pi)= \Pi^\prime$とを満たすとする.$T$全体のなす集合を$A_{\Pi,\Pi^\prime}$とする.
この時,
\begin{equation*}
 \mathrm{dist} (\Pi,\Pi^\prime):= \mathrm{inf}_{T \in A_{\Pi,\Pi^\prime}} ||T -I||
\end{equation*}
で定める.
$(DPL,dist)$は距離空間になる.

$D$-palne $\Pi$に対し,
\begin{equation*}
 \Psi :\bpd \to \Pi^{\perp}
\end{equation*}
を$C^{1,1}$級関数(定義不明)で$\Psi(0)=0$とする.
この時,a patch of radius $r$ over $\Pi$ centered at 0とは,
\begin{equation*}
 \Gamma=\{ x + \Psi(x) \mid x \in \bpd \}
\end{equation*}
をさす.さらに
\begin{equation*}
 \gnd := \mathrm{sup}_{x \neq y \in \bpd} \frac{ \nabla \Psi(x) - \nabla \Psi(y)}{||x - y||}
\end{equation*}
とする.ここで
$\nabla\Psi(x):\Pi \to \Pi^{\perp}$は接ベクトル空間の間の射$T_x\Pi \to T_x\Pi^{\perp}$のこと(としか考えられない).
ただし,$\Pi \sim T_x \Pi, \Pi^{\perp} \sim T_x\Pi^{\perp}$を使って同一視している.
実質$x$でのgradient.

もし$ \nabla \Psi(0)=0$(0写像)が成り立っている場合,$\Gamma$を
a patch of radius r tangent to $\Pi$ at its center 0という.

\begin{lem}
  $\Gamma_1$を$\Pi_1$上の半径$r_1$で中心$z_1$で$\Pi_1$に接しているpatchとする.
  $z_2 \in \Gamma_1$で$||z_2 -z_1 || < c_0r_1$を満たすとする.
  \begin{equation*}
   \gnaaad \le \frac{c_0}{r_1}
  \end{equation*}
  とする.$\Pi_2 \in DPL$で$\mathrm{dist}(\Pi_2,\Pi_1) <c_0$とする.
  この時,$\Pi_2$上の半径$c_1r_1$で中心$z_2$のpatch $\Gamma_2$で以下を満たすものが存在する.
  \begin{equation*}
   \gn{2}{0}{c_1r_1}{} \ge \frac{200c_0}{r_1}
  \end{equation*}
と
\begin{equation*}
\Gamma_2 \cap \bh{z_2}{\frac{c_1r_1}{2}} = \Gamma_1 \cap \bh{z_2}{\frac{c_1r_1}{2}}
\end{equation*}
\end{lem}
\begin{proof}
 わからん….恐らく陰関数定理を使いたいのだろうが,条件を満たす$\Gamma$の一意性が言えないような…
\end{proof}

\begin{dfn}
  $M \subset \mathcal{H}$が"compact imbedded D-manifold" (for short, just a "manifold")とは,以下が成り立つことである.
\begin{itemize}
  \item Mがcompact
  \item 任意の$z \in M$に対し,ある$T_zM \in DPL$と,$T_zM$上の半径$r_1$,中心$z$,$z$で$T_z(M)$
  に接するpatch$\Gamma$が存在し,
  $M \cap B_{\mathcal{H}}(z,r_2)=\Gamma \cap B_{\mathcal{H}}(z,r_2)$となる.
\end{itemize}
We say that $M$ has infinitesimal reach $\rho$ if for every $\rho^{\prime} < \rho$, there is a choice of $r_1 > r_2 > 0$ such that
for every $z \in M$ there is a patch $\Gamma$ over $T_z(M)$ of radius $r_1$, centered at $z$ and tangent to $T_z(M)$ at $z$ which has $C^{1,1}$-norm at most $1/\rho^{\prime}$
\end{dfn}

\begin{lem}[Growing Patch]
  Let $M$ be a manifold and let $r_1,r_2$ be as in the definition of a manifold.
Suppose $M$ has infinitesimal reach $ \ge 1$. Let $\Gamma \subset M$ be a patch of radius $r$ centered at 0, over $T_0M$. Suppose
$r$ is less than a small enough constant $\hat{c}$ determined by $D$. Then there exists a patch $\Gamma^{+}$ of radius $r + cr_2$
over $T_0M$, centered at 0 such that $\Gamma \subset \Gamma^{+} \subset M$
\end{lem}
\begin{proof}
 $\mathcal{H}$はヒルベルト空間なので,$\mathcal{H}=T_0M \oplus T_0M^{\perp}$に直和分解する.
 $M$が$D$次元なので,$T_0M \simeq \mathbb{R}^D$となる.
 patch $\Gamma$を
 $\Gamma =\{(x,\Psi(x) \mid x \in B_{\mathbb{R}^D}(0,r) \}$と書く,
ここで,$C^{1,1}$写像$\Psi: B_{\mathbb{R}^D}(0,r) \to \mathcal{H}^{\perp}$は
$\Psi(0)=0,\nabla\Psi(0)=0$となり,$||\Psi||_{C^{1,1}(B_{\mathbb{R}^D}(0,r))} \le C_0$となる.
$r$を十分小さくとり,$y \in B_{\mathbb{R}^D(0,r)}$に対し,
$|\nabla \Psi(y)| \le C_0||y|| \le C_0$とできる.この時,
上のlemmaより(本当はきちんと条件を確認したい.
$\Pi_1=\Pi_2$でとっているのと平行移動していること,そもそもこの取り方してるならもっと弱い補題でいけるのでは?)
\begin{equation*}
 \Psi_y:B_{\mathbb{R}^D}(y,c^{\prime} r_2) \to \mathcal{H}^{\perp}
\end{equation*}
で$||\nabla\Psi_y(z)||,||\Psi_y||_{C^{1,1}}$が有界で,
\begin{equation*}
 M \cap B_{\mathcal{H}}((y,\Psi(y)),c^{\prime\prime} r_2) =
 \{ (z,\Psi_y(z)) \mid z \in B_{\mathbb{R}^D}(y,c^{\prime} r_2)\}
 \cap B_{\mathcal{H}}((y,\Psi(y)),c^{\prime\prime} r_2)
\end{equation*}
\end{proof}

\part{Main theorem 1の証明}
Main theorem 1の証明の概略をのべ,いくつかを実際に示す.
\begin{rem}
1点は論理的に間違いがあり,証明できていない(と思う)
\end{rem}

\section{証明のフロー}
\label{sec:証明のフロー}
基本的な証明の方針を述べる.

\begin{itemize}
  \item JLの補題による確率の評価
  \item 証明したい,確率をfat shattering dimensinにより評価
  \item fat shatterig dimensionを特別な関数を用い,実際に評価し,定数による評価の作成
  \item 多様体の距離を用いて確率の評価を出す(条件つき)
  \item 今回の問題を示すのに必要な条件を満たすことをClaim 1を使う
\end{itemize}


\section{TOOLS FROM EMPIRICAL PROCESS}
\label{sec:TOOLS FROM EMPIRICAL PROCESS}
証明に使う用語を定義する.
\begin{dfn}[metric entropy]
 距離空間$(Y;\rho)$と$Z \subset Y$,$\eta \in \mathbb{R}$に対し,
 任意の$y \in Y$に対し,$\rho(y,z) <\eta$となる$z \in Z$が存在する時,$Y$を$\eta$-net of $Y$という.
 距離空間$X$上の確率測度$P$に対し,$F \subset \mathrm{Map}_{cont}(X,\mathbb{R})$は
 $L^2$-ノルムにより距離空間となる.(完備??)
 この時$N(\eta,F,L^2(P))$を$F$の$\eta$-netとなる有限集合$Y$の位数の最小値で定め,これを\textbf{metric entropy}という.
\end{dfn}

\begin{dfn}[Fat-shattering dimension]
  Let $F \subset \mathrm{Map}_{cont}(\mathbb{R},\mathbb{R}$.
  We say that a set of points $(x_1, \dots, x_k)$ is $\gamma$-shattered by $F$
  if there is $t = (t_1, \dots,t_k) \in \mathbb{R}^k$ such that for all
  $b = (b_1, \dots,b_k) \in {\pm 1}^k$, each $i \in [s]= \{1; \dots,s\}$,
   there is a function $f_{b,t}$ satisfying,
   \begin{equation*}
     b_i (f_{b,t}(xi) - t_i) >  \gamma
   \end{equation*}
  Fat-shattering dimension $\mathrm{Fat}_{\gamma}(F)$を$\gamma$-shattering setの位数の極大値で定める.
\end{dfn}
\begin{rem}
 $F$の取り方に依存する.$F$として$\mathrm{Map}_{cont}(\mathbb{R},\mathbb{R})$を取ると
 常にFat-shattering dimensionは$\infty$となる.
\end{rem}

\begin{dfn}[VC dimension]
Let $\Lambda$ be a collection of measurable subsets of $\mathbb{R}^m$.
For $x_1,\dots,x_k \in \mathbb{R}^m$,
let $\#\{(x1, \dots ,x_k) \cap H \mid H \in \Lambda\}$ be denoted the shatter coefficient $N(x_1, \dots, x_k)$.
The VC dimension $VC_{\Lambda}$
is the largest integer $k$ such that there exists $(x_1, \dots,x_k)$ such that $N(x_1,\dots, x_k) = 2^k$.
\end{dfn}
\begin{rem}
 $(x_1,\dots,x_k)$のうち欲しいものだけを含む$H$が必ず存在するのがVC dimension.
\end{rem}
VC dimensionには以下が成り立つ.
\begin{thm}
 Let $\Lambda$ be the class of halfspaces in $\mathbb{R}^g$. Then $VC= g + 1$.
\end{thm}
\begin{lem}
  For any $(x_1,\dots, x_k) \in  \mathbb{R}^g$,
  $N(x_1,\dots, x_k)  \le \sum_{i=0}^{VC_{\Lambda}}  \binom{k}{i}$
\end{lem}

Johson-Lindenstartusussの補題を述べる.

\begin{lem}[Johnson-Lindenstrauss]
  Let $ \epsilon \in (0, 1/2)$. Let $Q ⊂ \mathbb{R}^d$ be a set of $n$ points and $k =\frac{20 \mathrm{log}n}{\epsilon^2}$
  There exists a Lipshcitz mapping $f : \mathbb{R}^d \to \mathbb{R}^k$ such that for all $u, v \in Q$:
  \begin{equation*}
    (1 − \epsilon)||u − v||^2 \le ||f(u) − f(v)||^2 \le (1 + \epsilon )||u − v||^2
\end{equation*}
\end{lem}
なお,これを確率論に適用したものを本論文で使う.
\begin{lem}[JL確率版]
A related lemma is the distributional JL lemma. This lemma states that for any
$0 <\epsilon, \delta<1/2$ and positive integer $d$, there exists a distribution over $\mathbb{R}^{kd}$,
from which the matrix $A$ is drawn such that for $k = O(\epsilon−2log(1/\delta))$ and for any unit-length vector $x  \in \mathbb{R}^d$,
the claim below holds.
\begin{equation*}
  {\displaystyle P(|\Vert Ax\Vert _{2}^{2}-1|>\varepsilon )<\delta }
\end{equation*}
\end{lem}
One can obtain the JL lemma from the distributional version by setting
${\displaystyle x=(u-v)/\|u-v\|_{2}} x=(u-v)/\|u-v\|_{2}$ and ${\displaystyle \delta <1/n^{2}} \delta <1/n^{2}$
for some pair $u,v$ both in $X$.
Then the JL lemma follows by a union bound over all such pairs.

本論文ではそのcorollaryとして特に,以下を用いる.
\begin{lem}[JL]
  Let $y_1,\dots,y_l$ be points in the unit ball in $\mathbb{R}^m$ for some finite $m$.
Let $\mathbb{R}$ be an orthogonal projection onto a random $g$-dimensional subspace
(where $g = C \frac{\mathrm{log}}{\gamma^2}$) for some $\gamma > 0$ and an
absolute constant $C$). Then,
\begin{equation*}
\mathbb{P}[\mathrm{sup}_{i,j \in \{1,\dots,g\}}] | \frac{m}{g}Ry_i\cdot Ry_j -y_i\cdot y_j| > \frac{ \gamma}{2}] < \frac{1}{2}
\end{equation*}
\end{lem}

JLを使いfat-shattering dimensionを用いて確率の評価をしたのが以下である.
\begin{lem}
  Let $\mu$ be a measure supported on $X$, $F \subset \mathrm{Map}(X,\mathbb{R})$.
  Let $x_1,\dots, x_s$ be independent random variables drawn from $\mu$
  and $s$ be the uniform measure on $x := (x_1,\dots, x_s)$.
  If
\begin{equation*}
s \ge \frac{C}{\epsilon^2}(( \int_{c \epsilon}^{\infty}\sqrt{\mathrm{fat}_{\gamma}(F)}\mathrm{d}\gamma )^2 + \mathrm{log}1/\delta)
\end{equation*}
 then,
\begin{equation*}
 \mathbb{P}[ \mathrm{sup}_{f \in F} |\mathbb{E}_{\mu_s}f(x_i) -\mathbb{E}_{\mu}f|  \ge \epsilon] \le \delta.
\end{equation*}
\end{lem}
\begin{rem}
 定理の主張がおそらく,間違っていたので修正した(論文では$1- \delta$となっていたが,実際は$\delta$の間違いだと思う)
 でないと以降で評価できない.
\end{rem}

fat-shattering dimensionを評価する.
\begin{lem}
  Let $P$ be a probability distribution supported on $B_{\mathcal{H}}$.
  Let $F_{k,l} = \mathrm{Map}(B_{\mathcal{H}},\mathbb{R} \mid f(x)=\mathrm{max}_j \mathrm{min}_i(v_{ij} \cdot x) )$

\begin{align*}
   & \mathrm{fat}_{\gamma}(F_{k,l}) \le \frac{Ckl}{\gamma^2}\mathrm{log}^2 \frac{Ckl}{\gamma^2} \\
  \mbox{If } & s \ge \frac{C}{\epsilon^2 }(kl \mathrm{log}^4(kl/\epsilon^2)+ \mathrm{log}{1/\delta}), \\
  \mbox{then } & \mathbb{P}[\mathrm{sup}_{f \in F_{k,l}}| \mathbb{E}_{\mu_s}f(x_i) - \mathbb{E}_{\mu}f| \ge \epsilon ] \le 1 -\delta
\end{align*}
\end{lem}
\begin{proof}
  fat-shattering dimensionを上からboundする.
  $X=(x_1,\dots,x_s)(x_i \in B_{\mathcal{H}})$を$\gamma$-shatteredとする.
  この時,任意の$A \subset X$に対し,$f_A \in F_{k.l},t=(t_1,\dots,t_s)$が存在し,以下が成り立つ.
  \begin{eqnarray*}
 x_i \in A \mbox{の時 } f_A(x_i) - t_i > \gamma   \\
x_i \notin A \mbox{の時 } f_A(x_i) -t_i < - \gamma
  \end{eqnarray*}
  今後の都合に合わせ(本当に?)$X \setminus A$を$A$とする.つまり
  \begin{eqnarray*}
 x_i \in A \mbox{の時 } f_A(x_i) - t_i < - \gamma   \\
x_i \notin A \mbox{の時 } f_A(x_i) -t_i > \gamma
  \end{eqnarray*}
とする.
  $f_A$の定義から,ある$v_{ij} \in B$を用いて,
  $f_A(x)=\mathrm{max}_j \mathrm{min}_i(v_{ij} \cdot x) $とかける.
これを用いると,上の条件は以下のようになる.
\begin{eqnarray*}
x_r \in A \mbox{の時,任意の }j\mbox{に対し,ある}i{が存在し,} v_{ij} \cdot x_r  < t_r - \gamma   \\
x_r \in A \mbox{の時,ある }j\mbox{が存在し,任意の}i{に対し,} v_{ij} \cdot x_r  > t_r + \gamma   \\
\end{eqnarray*}

  Let $g= C_1(4\gamma^{-2} \mathrm{log}(s + kl))$
  for a sufficiently large universal(意味不明) constant $C_1$.
$g$次元部分空間 $V \subset \mathrm{span}(X cup V) $に対し,$ R : \mathrm{span}(X cup V) \to V$を自然な射影とする.
  この時JLより$y_i,y_j \in \{ x_1 ,\dots ,x_s,v_11 ,\dots ,v_{kl} \}$を用い
  \begin{equation*}
   \mathbb{P} [\mathrm{sup}_{i,j \in \{ 1,\dots, s+kl\} } | \frac{m}{g}(Ry_i)\cdot (Ry_j) - y_i \cdot  y_j| > \gamma ] < \frac{1}{2}
 \end{equation*}
となる.
余事象の確率とsupの定義から,
\begin{eqnarray*}
  \mathbb{P} [\mathrm{sup}_{i,j \in \{ 1,\dots, s+kl\} } | \frac{m}{g}(Ry_i)\cdot (Ry_j) - y_i \cdot  y_j |\le  \gamma ] \ge \frac{1}{2} \\
    \mathbb{P} [ | \frac{m}{g}(Rv_{ij})\cdot (Rx_r) - v_{ij} \cdot  x_r | \le  \gamma ] \ge \frac{1}{2}
\end{eqnarray*}
となる.

$x_r \in A$の時,任意の$j$に対し,ある$i$が存在し,以下が成り立つ.
\begin{eqnarray*}
  \mathbb{P} [ | \frac{m}{g}(Rv_{ij})\cdot (Rx_r) - v_{ij} \cdot  x_r \le  \gamma ]
  & \ge   \mathbb{P} [ | \frac{m}{g}(Rv_{ij})\cdot (Rx_r)| - |v_{ij} \cdot  x_r| \le  \gamma ] \\
  & \ge \mathbb{P} [ | \frac{m}{g}(Rv_{ij})\cdot (Rx_r)| - t_r + \gamma  \le  \gamma ] \\
  & = \mathbb{P} [ | \frac{m}{g}(Rv_{ij})\cdot (Rx_r)| \le t_r  ]
\end{eqnarray*}

$x_r \notin A$の時,ある$j$が存在し,任意の$i$に対し,いかが成り立つ.
\begin{eqnarray*}
  \mathbb{P} [ | \frac{m}{g}(Rv_{ij})\cdot (Rx_r) - v_{ij} \cdot  x_r \le  \gamma ]
  & \ge  & \mathbb{P} [ - | \frac{m}{g}(Rv_{ij})\cdot (Rx_r)| + |v_{ij} \cdot  x_r| \le  \gamma ] \\
  & \ge & \mathbb{P} [ - | \frac{m}{g}(Rv_{ij})\cdot (Rx_r)| +  t_r +  \gamma  \le  \gamma ] \\
  & =&  \mathbb{P} [ | \frac{m}{g}(Rv_{ij})\cdot (Rx_r)| \ge  t_r  ]
\end{eqnarray*}
$J:=\mathrm{Span}RX$とし,$t^J J \to \mathbb{R}$をある$i$が存在し,$y=Rx_i$と書ける時,
$t_j(y)$をその最小の$i$で定め,それ以外の時$0$とする.

これより,
\end{proof}




\section{A bound on the size of an $\epsilon$-net}
\label{a bound on the size of an epsilon net}
この章では、Claim 1と上の定理を用い、実際に主定理1を示す。

\begin{dfn}
 Let $(X,d)$ be a metric space, and $r > 0$. We call that $Y \subset X$ is an $r$-net of $X$
 if   for each $x \in X$, there is a $y \in Y$ such that $d(x,y) < r$.
\end{dfn}

\begin{cor} Let
  \begin{equation*}
   U_{\mathcal{G}}:\mathbb{R}^{+} \to \mathbb{R}
  \end{equation*}
be given by
\begin{equation*}
 U_{\mathcal{G}}(1/r) = CV ( \frac{ 1 }{\tau^d }  + \frac{ 1 }{ (\tau r)^{d/2} }).
\end{equation*}
Let $M \in \mathcal{G}$, and $M$ be equipped with the metric $d_{\mathcal{H}}$ of the $\mathcal{H}$.
Then, for any $r >0$, there is an $\sqrt{\tau r}$-net of $M$ consisting of no more than $U_{\mathcal{G}}(1/r)$ points.
\end{cor}
\begin{rem}
 なぜ、逆数で与える。違和感しか,感じない.
\end{rem}

\begin{proof}
  $\mathcal{G}$はコンパクトなので、$ s < r$とした時、$B(x,s)$を$x$を中心として半径$s$の開球とすると、$\cup_x B(x,s)$は有限部分被覆$\{ B(x_i.s) \}$を持つ。
  この時、$Y=\{ x_i \}$は有限で、任意の$x_i \in Y$に対し、$\mathrm{mini}|x_ i -x_j| < \epsilon$となる.
  $Y$が$M$の$r$-netになることは上より、明らか。

これもうまく証明できていない.
\end{proof}


\end{document}
